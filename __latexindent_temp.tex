@article{0004-637X-787-2-167,
  author={C.-Y. Ng and J. Takata and G. C. K. Leung and K. S. Cheng and P. Philippopoulos},
  title={High-energy Emission of the First Millisecond Pulsar},
  journal={The Astrophysical Journal},
  volume={787},
  number={2},
  pages={167},
  url={http://stacks.iop.org/0004-637X/787/i=2/a=167},
  year={2014},
  abstract={We report on X-ray and gamma-ray observations of the millisecond pulsar (MSP) B1937+21 taken with the Chandra X-ray Observatory , XMM-Newton , and the Fermi Large Area Telescope. The pulsar X-ray emission shows a purely non-thermal spectrum with a hard photon index of 0.9 ± 0.1, and is nearly 100% pulsed. We found no evidence of varying pulse profile with energy as previously claimed. We also analyzed 5.5 yr of Fermi survey data and obtained much improved constraints on the pulsar's timing and spectral properties in gamma-rays. The pulsed spectrum is adequately fitted by a simple power-law with a photon index of 2.38 ± 0.07. Both the gamma-ray and X-ray pulse profiles show similar two-peak structure and generally align with the radio peaks. We found that the aligned profiles and the hard spectrum in X-rays seem to be common properties among MSPs with high magnetic fields at the light cylinder. We discuss a possible physical scenario that could give rise to these features.}
}


@article{Tong2015,
  author={Tong, Hao},
  title={Pulsar braking: magnetodipole vs. wind},
  journal={Science China Physics, Mechanics Astronomy},
  volume={59},
  number={1},
  pages={619501},
  abstract={Pulsars are good clocks in the universe. One fundamental question is that why they are good clocks? This is related to the braking mechanism of pulsars. Nowadays pulsar timing is done with unprecedented accuracy. More pulsars have braking indices measured. The period derivative of intermittent pulsars and magnetars can vary by a factor of several. However, during pulsar studies, the magnetic dipole braking in vacuum is still often assumed. It is shown that the fundamental assumption of magnetic dipole braking (vacuum condition) does not exist and it is not consistent with the observations. The physical torque must consider the presence of the pulsar magnetosphere. Among various efforts, the wind braking model can explain many observations of pulsars and magnetars in a unified way. It is also consistent with the up-to-date observations. It is time for a paradigm shift in pulsar studies: from magnetic dipole braking to wind braking. As one alternative to the magnetospheric model, the fallback disk model is also discussed.},
  url={https://doi.org/10.1007/s11433-015-5752-x}
}

@article{PhysRevD.91.063007,
  title = {Braking index of isolated pulsars},
  author = {Hamil, O. and Stone, J. R. and Urbanec, M. and Urbancov\'a, G.},
  journal = {Phys. Rev. D},
  volume = {91},
  issue = {6},
  pages = {063007},
  numpages = {9},
  year = {2015},
  month = {Mar},
  publisher = {American Physical Society},
  doi = {10.1103/PhysRevD.91.063007},
  url = {https://link.aps.org/doi/10.1103/PhysRevD.91.063007}
}


@article{Sturrock:1971zc,
      author         = {Sturrock, P. A.},
      title          = {A Model of pulsars},
      journal        = {Astrophys. J.},
      volume         = {164},
      year           = {1971},
      pages          = {529},
      doi            = {10.1086/150865}
}

@article{0004-637X-603-1-283,
  author={Zakir F. Seidov},
  title={The Roche Problem: Some Analytics},
  journal={The Astrophysical Journal},
  volume={603},
  number={1},
  pages={283},
  url={http://stacks.iop.org/0004-637X/603/i=1/a=283},
  year={2004},
  abstract={Exact analytical formulae are derived for the potential and mass ratio as a function of Lagrangian point positions, in the classical Roche model of close binary stars.}
}


@article{0004-637X-744-1-33,
  author={L. Guillemot and T. J. Johnson and C. Venter and M. Kerr and B. Pancrazi and M. Livingstone and G. H. Janssen and P.
Jaroenjittichai and M. Kramer and I. Cognard and B. W. Stappers and A. K. Harding and F. Camilo and C. M. Espinoza and P. C. C.
Freire and F. Gargano and J. E. Grove and S. Johnston and P. F. Michelson and A. Noutsos and D. Parent and S. M. Ransom and P. S.
Ray and R. Shannon and D. A. Smith and G. Theureau and S. E. Thorsett and N. Webb},
  title={Pulsed Gamma Rays from the Original Millisecond and Black Widow Pulsars: A Case for Caustic Radio Emission?},
  journal={The Astrophysical Journal},
  volume={744},
  number={1},
  pages={33},
  url={http://stacks.iop.org/0004-637X/744/i=1/a=33},
  year={2012},
  abstract={We report the detection of pulsed gamma-ray emission from the fast millisecond pulsars (MSPs) B1937+21 (also known as J1939+2134) and B1957+20 (J1959+2048) using 18 months of survey data recorded by the Fermi Large Area Telescope and timing solutions based on radio observations conducted at the Westerbork and Nançay radio telescopes. In addition, we analyzed archival Rossi X-ray Timing Explorer and XMM-Newton X-ray data for the two MSPs, confirming the X-ray emission properties of PSR B1937+21 and finding evidence (~4σ) for pulsed emission from PSR B1957+20 for the first time. In both cases the gamma-ray emission profile is characterized by two peaks separated by half a rotation and are in close alignment with components observed in radio and X-rays. These two pulsars join PSRs J0034–0534 and J2214+3000 to form an emerging class of gamma-ray MSPs with phase-aligned peaks in different energy bands. The modeling of the radio and gamma-ray emission profiles suggests co-located emission regions in the outer magnetosphere.}
}

	
@article{0067-0049-208-2-17,
  author={A. A. Abdo and M. Ajello and A. Allafort and L. Baldini and J. Ballet and G. Barbiellini and M. G. Baring and D. Bastieri and A.
Belfiore and R. Bellazzini and B. Bhattacharyya and E. Bissaldi and E. D. Bloom and E. Bonamente and E. Bottacini and T. J.
Brandt and J. Bregeon and M. Brigida and P. Bruel and R. Buehler and M. Burgay and T. H. Burnett and G. Busetto and S. Buson and G. A.
Caliandro and R. A. Cameron and F. Camilo and P. A. Caraveo and J. M. Casandjian and C. Cecchi and Ö. Çelik and E. Charles and S.
Chaty and R. C. G. Chaves and A. Chekhtman and A. W. Chen and J. Chiang and G. Chiaro and S. Ciprini and R. Claus and I. Cognard and J.
Cohen-Tanugi and L. R. Cominsky and J. Conrad and S. Cutini and F. D'Ammando and A. de Angelis and M. E. DeCesar and A. De
Luca and P. R. den Hartog and F. de Palma and C. D. Dermer and G. Desvignes and S. W. Digel and L. Di Venere and P. S. Drell and A.
Drlica-Wagner and R. Dubois and D. Dumora and C. M. Espinoza and L. Falletti and C. Favuzzi and E. C. Ferrara and W. B. Focke and A.
Franckowiak and P. C. C. Freire and S. Funk and P. Fusco and F. Gargano and D. Gasparrini and S. Germani and N. Giglietto and P.
Giommi and F. Giordano and M. Giroletti and T. Glanzman and G. Godfrey and E. V. Gotthelf and I. A. Grenier and M.-H. Grondin and J.
E. Grove and L. Guillemot and S. Guiriec and D. Hadasch and Y. Hanabata and A. K. Harding and M. Hayashida and E. Hays and J.
Hessels and J. Hewitt and A. B. Hill and D. Horan and X. Hou and R. E. Hughes and M. S. Jackson and G. H. Janssen and T. Jogler and G.
Jóhannesson and R. P. Johnson and A. S. Johnson and T. J. Johnson and W. N. Johnson and S. Johnston and T. Kamae and J.
Kataoka and M. Keith and M. Kerr and J. Knödlseder and M. Kramer and M. Kuss and J. Lande and S. Larsson and L. Latronico and M.
Lemoine-Goumard and F. Longo and F. Loparco and M. N. Lovellette and P. Lubrano and A. G. Lyne and R. N. Manchester and M.
Marelli and F. Massaro and M. Mayer and M. N. Mazziotta and J. E. McEnery and M. A. McLaughlin and J. Mehault and P. F.
Michelson and R. P. Mignani and W. Mitthumsiri and T. Mizuno and A. A. Moiseev and M. E. Monzani and A. Morselli and I. V.
Moskalenko and S. Murgia and T. Nakamori and R. Nemmen and E. Nuss and M. Ohno and T. Ohsugi and M. Orienti and E. Orlando and J. F.
Ormes and D. Paneque and J. H. Panetta and D. Parent and J. S. Perkins and M. Pesce-Rollins and M. Pierbattista and F. Piron and G.
Pivato and H. J. Pletsch and T. A. Porter and A. Possenti and S. Rainò and R. Rando and S. M. Ransom and P. S. Ray and M. Razzano and N.
Rea and A. Reimer and O. Reimer and N. Renault and T. Reposeur and S. Ritz and R. W. Romani and M. Roth and R. Rousseau and J. Roy and J.
Ruan and A. Sartori and P. M. Saz Parkinson and J. D. Scargle and A. Schulz and C. Sgrò and R. Shannon and E. J. Siskind and D. A.
Smith and G. Spandre and P. Spinelli and B. W. Stappers and A. W. Strong and D. J. Suson and H. Takahashi and J. G. Thayer and J. B.
Thayer and G. Theureau and D. J. Thompson and S. E. Thorsett and L. Tibaldo and O. Tibolla and M. Tinivella and D. F. Torres and G.
Tosti and E. Troja and Y. Uchiyama and T. L. Usher and J. Vandenbroucke and V. Vasileiou and C. Venter and G. Vianello and V.
Vitale and N. Wang and P. Weltevrede and B. L. Winer and M. T. Wolff and D. L. Wood and K. S. Wood and M. Wood and Z. Yang},
  title={The Second Fermi Large Area Telescope Catalog of Gamma-Ray Pulsars},
  journal={The Astrophysical Journal Supplement Series},
  volume={208},
  number={2},
  pages={17},
  url={http://stacks.iop.org/0067-0049/208/i=2/a=17},
  year={2013},
  abstract={This catalog summarizes 117 high-confidence ≥0.1 GeV gamma-ray pulsar detections using three years of data acquired by the Large Area Telescope (LAT) on the Fermi satellite. Half are neutron stars discovered using LAT data through periodicity searches in gamma-ray and radio data around LAT unassociated source positions. The 117 pulsars are evenly divided into three groups: millisecond pulsars, young radio-loud pulsars, and young radio-quiet pulsars. We characterize the pulse profiles and energy spectra and derive luminosities when distance information exists. Spectral analysis of the off-peak phase intervals indicates probable pulsar wind nebula emission for four pulsars, and off-peak magnetospheric emission for several young and millisecond pulsars. We compare the gamma-ray properties with those in the radio, optical, and X-ray bands. We provide flux limits for pulsars with no observed gamma-ray emission, highlighting a small number of gamma-faint, radio-loud pulsars. The large, varied gamma-ray pulsar sample constrains emission models. Fermi 's selection biases complement those of radio surveys, enhancing comparisons with predicted population distributions.}
}

@article{0004-637X-720-1-178,
  author={Y. Wang and J. Takata and K. S. Cheng},
  title={Gamma-ray Spectral Properties of Mature Pulsars: A Two-layer Model},
  journal={The Astrophysical Journal},
  volume={720},
  number={1},
  pages={178},
  url={http://stacks.iop.org/0004-637X/720/i=1/a=178},
  year={2010},
  abstract={We use a simple two-layer outer gap model, whose accelerator consists of a primary region and a screening region, to discuss the γ-ray spectrum of mature pulsars detected by Fermi . By solving the Poisson equation with an assumed simple step-function distribution for the charge density in these two regions, the distribution of the electric field and the curvature radiation process of the accelerated particles can be calculated. In our model, the properties of the phase-averaged spectrum can be completely specified by three gap parameters, i.e., the fractional gap size in the outer magnetosphere, the gap current in the primary region, and the gap size ratio between the primary region and the total gap size. We discuss how these parameters affect the spectral properties. We argue that although the radiation mechanism in the outer gap is a curvature radiation process, the observed gamma-ray spectrum can substantially deviate from the simple curvature spectrum because the overall spectrum consists of two components, i.e., a primary region and a screening region. In some pulsars, the radiation from the screening region is so strong that the photon index from 100 MeV to several GeV can be as flat as ~2. We show that the fitting fractional gap thickness of the canonical pulsars increases with the spin-down age. We find that the total gap current is about 50% of the Goldreich-Julian value and the thickness of the screening region is a few percent of the total gap thickness. We also find that the predicted γ-ray luminosity is less dependent on the spin-down power ( L sd ) for pulsars with L sd ##IMG## [http://ej.iop.org/icons/Entities/gsim.gif] {gsim} 10 36  erg s –1 , while the γ-ray luminosity decreases with the spin-down power for pulsars with L sd ##IMG## [http://ej.iop.org/icons/Entities/lsim.gif] {lsim} 10 36 erg s –1 . This relation may imply that the major gap closure mechanism is a photon-photon pair-creation process for pulsars with L sd ##IMG## [http://ej.iop.org/icons/Entities/gsim.gif] {gsim} 10 36 erg s –1 , while it is a magnetic pair-creation process for pulsars with L sd ##IMG## [http://ej.iop.org/icons/Entities/lsim.gif] {lsim} 10 36 erg s –1 .}
}

@article{0004-637X-845-2-159,
  author={E. V. Gotthelf and S. Bogdanov},
  title={NuSTAR Hard X-Ray Observations of the Energetic Millisecond Pulsars PSR B1821-24, PSR B1937+21, and PSR J0218+4232},
  journal={The Astrophysical Journal},
  volume={845},
  number={2},
  pages={159},
  url={http://stacks.iop.org/0004-637X/845/i=2/a=159},
  year={2017},
  abstract={We present Nuclear Spectroscopic Telescope Array ( NuSTAR ) hard X-ray timing and spectroscopy of the three exceptionally energetic rotation-powered millisecond pulsars PSRs B1821-24, B1937+21, and J0218+4232. By correcting for the frequency and phase drifts of the NuSTAR onboard clock, we are able to recover the intrinsic hard X-ray pulse profiles of all three pulsars with a resolution down to ##IMG## [http://ej.iop.org/images/0004-637X/845/2/159/apjaa813cieqn1.gif] {$\leqslant 15\,\mu {\rm{s}}$} . The substantial reduction of background emission relative to previous broadband X-ray observations allows us to detect for the first time pulsed emission up to ∼50 keV, ∼20 keV, and ∼25 keV for the three pulsars, respectively. We conduct phase-resolved spectroscopy in the 0.5–79 keV range for all three objects, obtaining the best measurements yet of the broadband spectral shape and high-energy pulsed emission to date. We find extensions of the same power-law continua seen at lower energies, with no conclusive evidence for a spectral turnover or break. Extrapolation of the X-ray power-law spectrum to higher energies reveals that a turnover in the 100 keV to 100 MeV range is required to accommodate the high-energy γ -ray emission observed with Fermi -LAT, similar to the spectral energy distribution observed for the Crab pulsar.}
}

@article{doi:10.1111/j.1365-2966.2011.18577.x,
  author = {Wang, Y. and Takata, J. and Cheng, K. S.},
  title = {Three-dimensional two-layer outer gap model: Fermi energy-dependent light curves of the Vela pulsar},
  journal = {Monthly Notices of the Royal Astronomical Society},
  volume = {414},
  number = {3},
  pages = {2664-2673},
  year = {2011},
  doi = {10.1111/j.1365-2966.2011.18577.x},
  URL = {http://dx.doi.org/10.1111/j.1365-2966.2011.18577.x},
  eprint = {/oup/backfile/content_public/journal/mnras/414/3/10.1111/j.1365-2966.2011.18577.x/2/mnras0414-2664.pdf}
}
	

@article{1969ApJ,
   author = {Goldreich, P. and Julian, W.~H.},
    title = {Pulsar Electrodynamics},
  journal = {The Astrophysical Journal},
     year = {1969},
    month = {aug},
   volume = {157},
    pages = {869},
   URL = {http://adsabs.harvard.edu/abs/1969ApJ...157..869G},
}

