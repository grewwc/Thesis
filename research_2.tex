\documentclass[12pt]{report}
\usepackage{geometry}	
\usepackage{cite}
\usepackage[utf8]{inputenc}
\usepackage{amsmath}
\usepackage{multicol}
\usepackage{titlesec}
\usepackage{graphicx}
\usepackage{wrapfig}
\usepackage{textcomp}
\usepackage{caption}
\usepackage{subcaption}
\usepackage{comment}
\usepackage{etoolbox}
\usepackage{anyfontsize}
\usepackage{url}
\usepackage{multirow}
\usepackage{array}
\usepackage{tabu}
\usepackage{hyperref}
\usepackage{color}
\usepackage{epigraph}
\usepackage{makebox}
\usepackage{graphicx}
\usepackage{array}
\usepackage{setspace}
\usepackage[english]{babel}


\renewcommand{\arraystretch}{2}
\renewcommand{\bibname}{REFERENCES}
\hypersetup{
    colorlinks,
    citecolor=black,
    filecolor=black,
    %linkcolor=[RGB]{0,204,0},
    linkcolor=black,
    urlcolor=black
}
% \captionsetup[figure]{labelfont=bf, font=footnotesize}

\setlength{\parskip}{1em}
% \setcounter{tocdepth}{4} 
% \setcounter{secnumdepth}{4}

%this several lines is for: no number before suction. (This is a bug)
\makeatletter
\patchcmd{\ttlh@hang}{\parindent\z@}{\parindent\z@\leavevmode}{}{}
\patchcmd{\ttlh@hang}{\noindent}{}{}{}
\makeatother


\geometry{
	a4paper,
	% total={210mm,297mm},
 	left=35mm,
 	top=25mm,
 	right=35mm,
}

\title{\textbf{Broadband Spectra Analysis of Three Energetic Millisecond Pulsars}\\ \vspace{1cm}
			{\large Department of Physics, The University of Hong Kong, Pokfulam Road, Hong Kong}\\ \vspace{1cm}
			% {\includegraphics[scale=0.2]{{/Users/grewwc/Desktop/Thesis/hku.png}}}\\ \vspace{3cm}
}
\date{}
\author{Wang Wenchao  \\3030053350}
\setlength{\columnsep}{1cm}

% \titleformat
% {\chapter} % command
% [display] % shape
% {\bfseries\Large} % format
% {\textit{Chapter \thechapter}} % label
% {0.5ex} % sep
% {
%     \rule{\textwidth}{1pt}
%     \vspace{1ex}
%     \centering
% } % before-code

% \titleformat{\section}[hang]
% {\large\bfseries}
% {\thesection}
% {0.5em}
% {}

% \titleformat{\subsection}[hang]
% {\fontsize{12}{15}\bfseries\sffamily}
% {\thesubsection}
% {1em}
% {}

% \titleformat{\subsubsection}[hang]
% {\fontsize{11}{15}\bfseries\sffamily}
% {\thesubsection}
% {0.5em}
% {}

% \titleformat{\subsubsubsection}[hang]
% {\fontsize{9}{15}\bfseries\sffamily}
% {\thesubsection}
% {0.5em}
% {}

\newcommand{\mycaption}[1]{\protect \caption{#1}}
%Below is the main content.

%insert a single figure.
%parameters: 
%%  1. path
%%  2. scale
%%  3. caption
\newcommand{\singleFig}[3]{
  \begin{figure}[!htp]
    \centering
    \includegraphics[scale=#2]{/Users/grewwc/Desktop/Thesis/#1}
    \caption{#3}
    \label{fig: #1}
  \end{figure}
}

\newcommand{\gj}[0]{
  Goldreich-Julian charge density
}

\newcommand{\fgl}[0]{
  Fermi Lat four-year Point Source Catalog
}

\newcommand{\question}[1]{
  $<$\textbf{question}$>$#1$<$\textbf{/question}$>$
}

\newcommand{\change}[1]{
  $<$\colorbox{red}{\textbf{change}}$>$#1$<$\colorbox{red}{\textbf{/change}}$>$
}

\newcommand{\add}[1]{
  $<$\colorbox{red}{\textbf{add}}$>$#1$<$\colorbox{red}{\textbf{/add}}$>$
}

\newcommand{\mayAdd}[1]{
  $<$\colorbox{red}{\textbf{mayAdd}}$>$#1$<$\colorbox{red}{\textbf{/mayAdd}}$>$
}


\newcommand{\mayChange}[1]{
  $<$\colorbox{red}{\textbf{mayChange}}$>$#1$<$\colorbox{red}{\textbf{/mayChange}}$>$
}

\newcommand{\myComment}[1]{
  %#1 
  \newline
}

\newcommand{\Notice}[1]{
  $<$\textbf{Notice}$>$#1$<$\textbf{/Notice}$>$
}

\newcommand{\blackhref}[2]{
  \href{#1}{\color{black}{\textit{\small #2}}}
}



\begin{document}
\maketitle

\pagenumbering{roman}


% \thispagestyle{empty}
\cleardoublepage
\addcontentsline{toc}{chapter}{Acknowledgements}
\chapter*{Acknowledgements}
  This is acknowledgement. 
  

\cleardoublepage
\addcontentsline{toc}{chapter}{Abstract}
\chapter*{Abstract}
  \doublespacing
  In the thesis, I mainly introduce our study on the high energy spectra of three 
  millisecond pulsars which are PSR J0218+4232, PSR B1821-24 and PSR B1937+21. 
  The \textit{Fermi Lat} Pass 8 data was published in 2015 and has lots of advantages over 
  the old Pass 7 data, such as increased effective area and wider energy range. Since 
  the recent gamma-ray spectra analysis of the three MSPs are relatively old (in about 2014), 
  I redo the gamma-ray spectra analysis of the MSPs with 
  4-year more \textit{Fermi Lat} observation data and newly published Pass 8 data. 
  As expected, I obtain better fit results for gamma-ray spectra of 
  the three MSPs with smaller errors and larger test statistic values. Then I briefly 
  introduce a pulsar emission model called two-layer model \cite{0004-637X-787-2-167} and 
  do numerical simulation to test the two-layer model using the new observation data.
  By minimizing the differences between the predictions of the two-layer model and the real 
  data, I fit the independent parameters of the two-layer model, which is helpful to
  understand emission mechanisms of pulsars. I find that though the two-layer model is 
  simple, it can generate broad-band spectra of pulsars which are very close to the 
  observation data from \textit{Fermi Lat} in most energy bands.
			
\cleardoublepage
\addcontentsline{toc}{chapter}{\listfigurename}
\newpage

\cleardoublepage
\addcontentsline{toc}{chapter}{\listtablename}

\tableofcontents

\pagenumbering{arabic}

\listoffigures

\listoftables


\chapter{Introduction}   	   
  \section{Neutron Stars and Pulsars}
    Neutron stars are produced by supernovae explosion of massive stars which have about four 
    to eight solar mass. After a supernova explosion, a star leaves a central region. 
    And the central region collapses because of the effect of 
    gravity until protons and electrons combine to form neutrons 
    ($e^{-}+p\rightarrow n+\nu_{e}$) ---the reason why they are called ``neutron stars''.  
    Because neutrons have no electromagnetic force on each other, they can be squeezed very 
    tightly. Therefore, a neutron star has tremendous high density 
    (about $5\times 10^{17} \mbox{kg/m}^3$) and its diameter and mass is about20km and 
    1.4 solar mass respectively. What
    prevents a neutron star to continue to contract is the degeneracy pressure of neutrons. 
    
    Pulsars are fast-spinning neutron stars. They have rotational periods from a few 
    milliseconds to several seconds. For example, the rotational period of PSR B1937+21 is 
    about 1.56ms while PSR B1919+21 is approximately 1.34s. As we know, a star can be ripped 
    by centrifugal force if the star rotates too fast. We can estimate lower limit of 
    density of a star with the equation $\rho=\frac{3\pi}{P^2G}$, where $P$ is the 
    rotational period of a pulsar. Just for simplicity, let $P$ be 1s. Then we 
    get $\rho\approx 1.4\times 10^{11}\mbox{kg/m}^3$. With the knowledge that the 
    density of a white dwarf is about $1\times 10^9\mbox{kg/m}^3$ which is smaller than 
    the lower density limit, the observed fast-spinning stars belong to the kind of stars 
    which are much denser than white dwarf. As a result, neutron stars are ideal candidates 
    for pulsars. 
    
    More than 2000 pulsars have been discovered so far. Most of them are in the disk of 
    our Galaxy while we also can find a small portion of them in high latitude, which can be 
    seen clearly in Figure \ref{fig: spatial_distribution}. This may 
    because they cannot escape the gravitational potential if their kinetic energy is not 
    large enough. Besides, even though they have large enough velocities to escape from their 
    birth region, there are some probabilities that they become nearly non-detectable before 
    reaching high latitude. 

    \begin{figure}[!htp]
      \centering
      \includegraphics[scale=0.35]{/Users/grewwc/Desktop/Thesis/pulsar_distribution.png}
      \caption{Spatial distribution of some pulsars in galactic coordinate system.}
      \label{fig: spatial_distribution}
    \end{figure}
    
    \section{Emission Mechanism of Pulsars}
      Although the emission mechanism of pulsars has not been fully understood yet, some 
      models have been developed 
      trying to explain observational data. The following is a simple model that can explain 
      some basic features of pulsars spectra. I will first introduce the magnetic dipole model, 
      then the synchrotron radiation and inverse Compton radiation. 
      
      \subsection{Magnetic Dipole Model}
        Assuming a pulsar has a magnetic dipole moment $\vec{m}$, the angel between rotation axis and 
        direction of 
        $\vec{m}$ is $\alpha$, its angular velocity is $\Omega$, radius is R and moment of inertia is $I$. 
        Also assuming that energy of electromagnetic radiation is completely from the rotational energy, 
        its spin-down rate can be written as: 
        $$
            \dot{\Omega}=-\frac{B_p^2 R^6 \Omega^3 \sin{\alpha}^3}{6c^3I}
        $$
        where $B_p$ is magnetic field strength in the pole of the pulsar. Its surface magnetic field can 
        also be estimated by:
        $$
            B_s=\sqrt{\frac{3c^3I}{2\pi^2R^6}P\dot{P}}=3.2\times 10^{19}\sqrt{P\dot{P}}
        $$
        where $B_s$ is the strength of surface magnetic field. \\
        \indent In general, a pulsar's spin down rate can be expressed as: $\dot{\Omega}=-K\Omega^{n}$, 
        where K is a constant and n is called braking index. In magnetic dipole model n is 
        3 \cite{Tong2015}. Then characteristic age of the pulsar can be defined as: 
        $P/2\dot{P}$ in magnetic dipole model. For example, the Crab 
        pulsar's rotation period is about $0.033s$ and period derivative is 
        $4.22\times 10^{-13}s/s$. The characteristic age is about 1200 years. The pulsar 
        is remnant of a supernova which is observed by ancient astronomers in 1054 
        AD, so the record shows that characteristic age can give us and an order of magnetic 
        estimate of a pulsar's real age. \\
        \indent 
        Although the braking index is 3 in magnetic dipole model, most of pulsars' braking 
        index is less than 3 as shown in Figure \ref{fig:braking_index}. The reason is that if
        a pulsar's spin down is completely because of pulsar wind, the braking index is 1. 
        Thus, the real braking index should be a combination of 1 and 3, which is usually 
        less than 3. \cite{PhysRevD.91.063007}
        
        \begin{figure}[!ht]
          \centering
          \includegraphics[scale=0.6]{/Users/grewwc/Desktop/Thesis/table.png}
          \caption{Braking index of some pulsars.}
          \label{fig:braking_index}
        \end{figure}
    
        \subsection{Synchrotron Radiation}
          Synchrotron radiation is a special case of cyclotron radiation when a particle's
          speed is comparable to the speed of light. Because of the relativistic beaming 
          effect, a very short radiation pulse can be observed when speeds of particles are
          large. I only aim to analyze the spectral properties of MSPs, so I focus on the 
          spectrum property of synchrotron radiation. 
          With Larmor's Formula we can derive the synchrotron radiation power of an electron:
          \begin{equation}
            \label{func: sync_total_power_precise}
            P = \frac{2e^4\gamma^2\beta^2B_{\perp}^2}{3m_e^2c^3} 
          \end{equation}
          where $\gamma$ is the Lorentz factor of the electron, $\beta=v/c$ and $B_{\perp}$ 
          is the strength of magnetic field perpendicular to the circular motion plane. 
          When $\beta \sim 1$, Function \ref{func: sync_total_power_precise} can be 
          simplified as:
          \begin{equation}
            \label{func: sync_total_power_simplified}
            P = \frac{2}{3}\frac{e^2c}{R^2}\gamma^4
          \end{equation}
          where $R = E / e B_{\perp}$ is the radius of the electron's circular motion. 
          Furthermore, the power spectrum of a single electron  
          can be described by Function \ref{func: syncrothron spectrum}
          \begin{eqnarray}
          \label{func: syncrothron spectrum}
            P\left(\nu\right) &=& \frac{\sqrt{3} e^3 B \sin{\alpha}}{m c^2} 
              \left(\frac{\nu}{\nu_c}\right) \int_{\nu / \nu_c}^{\infty} K_{5/3}\left(\eta \right)d\eta  \nonumber \\
              &=& \frac{\sqrt{3}e^2}{m_eR}\gamma \left(\frac{\nu}{\nu_c}\right) \int_{\nu / \nu_c}^{\infty} K_{5/3}\left(\eta \right)d\eta
          \end{eqnarray}
          where $\nu_c$ is the critical frequency and $K_{5/3}$ is modified Bessel function. 
          The critical frequency can be expressed by Function \ref{func: critical_frequency}
          \begin{eqnarray}
            \label{func: critical_frequency}
            \nu_c &=& \frac{3}{2} \gamma^2 \nu_{cyc} \sin{\alpha} \nonumber \\
                  &=& \frac{3}{4\pi} \frac{c}{R} \gamma^3 
          \end{eqnarray} 
          where $\alpha$ is the pitch angle and the $\nu_{cyc}$ is the frequency of 
          corresponding cyclotron radiation. These functions do not give us very much 
          information because of the integration of the modified Bessel function. We let 
          $x = \nu / \nu_c$ and fix the environment variables such as magnetic field ($B$), 
          Function \ref{func: syncrothron spectrum} becomes: 
          \begin{equation}
            \label{func: to_x}
            P\left(\nu\right) = C \times x \int_{x}^{\infty} K_{5/3}\left(\eta \right)d\eta 
          \end{equation}
          where $C$ is a constant dependent on $B$. Thus, in order to analyze the power spectrum of synchrotron radiation,
          we only concentrate on the later part, which is
          \begin{equation}
            \label{func: fx}
            F\left(x\right) = x \int_{x}^{\infty} K_{5/3}\left(\eta \right)d\eta 
          \end{equation}
            
            % \begin{figure}[!ht] 
            % \begin{minipage}{\textwidth}
            %   \begin{center} 
            %     \includegraphics[scale=0.6]{/Users/grewwc/Desktop/Thesis/sync_spectrum_loglog.png}
            %   \end{center}
            %   \end{minipage}
            %   \\
            % \begin{minipage}{\textwidth}
            %   \begin{center}
            %   \includegraphics[scale=0.61]{/Users/grewwc/Desktop/Thesis/sync_power_loglog.png}
            %   \end{center}
            % \end{minipage}
            % \centering
            %   \mycaption{According to Function \ref{func: fx}, top: $F\left(x\right)$; bottom: $x F\left(x\right)$.
            %     \change{should change the style of .}}
            %   \label{fig: sync_spectrum_loglog}
            % \end{figure}
            
            \singleFig{sync_spectrum_loglog_combined}{0.5}{Spectrum shape of synchrotron 
            radiation for a single particle (top). According ot Equation \ref{func: fx},
            The top figure \ref{func: fx} describes the general shape of power spectrum of 
            synchrotron radiation. When the frequency is larger than
            the critical frequency $\nu_c$, the power goes down dramatically. However, the 
            top figure does not show the information that at what frequency the charged 
            particle emit the strongest power, which can be described in the bottom figure. 
            The bottom figure reveals that most energy is emitted around critical frequency.}
  
            In reality, synchrotron radiation is not generated by a single particle. We 
            describe the number density distribution of electrons with respect to energy by a 
            single power-law model:
            \begin{equation}
              \label{func: sync_number_density}
              N\left(E\right) \approx C E^{-\delta}
            \end{equation}
          
            For simplicity, we set the ambient magnetic field $B$ to be a constant and make an 
            approximation that all radiations are at a single frequency:
            \begin{equation}
              \label{func: sync_approximation}
              \nu \approx \gamma^2 \nu_{cyc}
            \end{equation}
            where the meaning of $\nu_{cyc}$ is the same as Function 
            \ref{func: critical_frequency}. Our objective is to know the relationship between 
            total power of all electrons and their radiation frequency. We describe the 
            relationship as Equation \ref{func: sync_power_single_frequency}
            
            \begin{eqnarray}
              \label{func: sync_power_single_frequency}
              -P\left(E\right)N\left(E\right)dE &=& Q_{\nu} d\nu\\
              P\left(E\right) &=& \frac{4}{3} \sigma_{T} \beta^2 \gamma^2 c U_B
            \end{eqnarray} 
            where $\sigma_{T}$ is electron Thompson scattering section, $U_B$ is energy 
            density of the environment magnetic field, $Q_{\nu}$ is the emission coefficient 
            of synchrotron radiation and $E=\gamma m_e c^2$. With Equation 
            \ref{func: sync_approximation}, we have
            \begin{equation}
              \label{func: sync_combine}
              P = \frac{dE}{d\nu} = \frac{m_e c^2}{2\sqrt{\nu \nu_{cyc}}}
            \end{equation}
            Combine Functions \ref{func: sync_combine} and 
            \ref{func: sync_power_single_frequency} we get:
            \begin{equation}
              Q_{\nu} = \frac{4}{3} \sigma_{T} \beta^2 \gamma^2 c U_B \frac{m_e c^2}{2\sqrt{\nu \nu_{cyc}}} N\left(E\right)
            \end{equation}
            Ignoring constants in Function \ref{func: sync_power_single_frequency} we have 
            \begin{equation}
              \label{func: sync_final}
              Q_{\nu} \propto \nu^{(1-\delta)/2}
            \end{equation}
            Function \ref{func: sync_final} shows that if the number density of electrons is 
            a power-law distribution, the spectrum of synchrotron radiation is also a 
            power-lay model.  
            
          \subsection{Inverse-Compton radiation}
            If an energetic relativistic photon collides with a charged particle from a 
            proper incident angle, the photon's energy decreases and its direction changes. 
            This is the process of Compton Scattering. Inverse-Compton radiation is the 
            opposite process and a low energy photon gained energy from an ultra-relativistic 
            electron in the process. 

            \singleFig{inverse_compton}{0.45}{Inverse Compton Diagram}
            As Figure \ref{fig: inverse_compton} shows, in the laboratory frame ($S$), the 
            incident angle and energy of a photon is $\theta$ and $h \nu$ respectively. 
            The speed of the electron is $v$. In the electron rest frame ($S^{\prime}$), 
            we change the denotation to $\theta^{\prime}$, $h \nu^{\prime}$ and. Also, let 
            the position of the electron be the origin point of $S^{\prime}$. We can study 
            the whole process in the $S^{\prime}$ frame, the transfer the result to the $S$ 
            frame by Lorentz transformation. 

            In the $S^{\prime}$ frame, the electron is at rest so its energy is $m_e c^2$. 
            For Inverse Compton scattering, the energy of an incident photon 
            (less than several $keV$) is much less than the rest energy of an electron 
            (about $0.51MeV$) and the relationship can be expressed by 
            $h\nu^{\prime} \ll m_e c^2$. Therefore, this can be treated as Thompson 
            Scattering process. Let the Poynting vector of incident photons be 
            $\vec{S}^{\prime}$ and their energy density be $U_{rad}^{\prime}$, we have 
            equation \ref{eq: poynting_and_energy_density}
            \begin{equation}
              \label{eq: poynting_and_energy_density}
              \vec{S}^{\prime} = c U_{rad}^{\prime}
            \end{equation}
            The electron absorbs the energy of the incident photons and then is accelerated. 
            Thus the accelerated electron will emit part of the energy taken from the incoming 
            photon and the power of scattered radiation is denoted as $P^{\prime}$.
            The ratio can be described by Thompson Scattering cross section $\sigma_{T}$ 
            which is:
            \begin{equation}
              \label{eq: thompson_cross_section}
              \sigma_{T} = \frac{8\pi}{3} \left(\frac{e^2}{m_e c^2}\right)^2
            \end{equation}
            and the relationship between the electron radiation power and incoming photon 
            energy flux can be described by Equation \ref{eq: relationship_power_poynting}
            \begin{equation}
              \label{eq: relationship_power_poynting}
              P^{\prime} = \left| \vec{S}^{\prime} \right| \sigma_{T}
            \end{equation}
            Combine Equations \ref{eq: poynting_and_energy_density} and 
            \ref{eq: relationship_power_poynting}, the radiation power emitted by the 
            electron is: 
            \begin{equation}
              \label{eq: final_relationship}
              P^{\prime} = c \sigma_{T} U^{\prime}_{rad}
            \end{equation}

            Then we need to find the relationship between the two frames $S$ and $S^{\prime}$. 
            It mainly contains two parts: the relationship between $P$, $P^{\prime}$ and 
            $U_{rad}$, $U_{rad}^{\prime}$. Since $P = dE/dt$ and it is Lorentz invariant 
            in inertial frame, we get the equation: 
            \begin{equation}
              \label{eq: power_is_equal}
              P = P^{\prime}
            \end{equation}
            Now we hope to write $U_{rad}^{\prime}$ in terms of $U_{rad}$. $U_{rad}$ is 
            comprised by energy of a single photon and photon density. In the following
            analysis, all the denotations are correspondent to Equation \ref{fig: inverse_compton}. 
            According to the relativistic Doppler shift formula, we have: 
            \begin{equation}
              \label{eq: doppler_shift}
              h \nu^{\prime} = \left(h \nu\right) \gamma \left(1 + \beta \cos{\theta} \right)
            \end{equation}
            where $\beta = v / c$ and $\gamma$ is Lorentz factor of an ultra-relativistic 
            electron. Then we calculate the photon density. In the frame $S^{\prime}$, the 
            photon density is inverse proportional to the time interval ($\Delta t$) between 
            the two photons striking the electron since total number of photons is Lorentz 
            invariant. In laboratory frame $S$, we consider two photons collide with the 
            electron at the 4-dimension vector of 
            $\left(x_{1}, 0, 0, t_{1}\right)$ and $\left(x_{2}, 0, 0, t_{2}\right)$. 
            According to the Lorentz transformation between inertial frames: 
            \begin{equation}
              \label{eq: lorentz_transfer_general}
                \begin{cases}
                  & x = \gamma \left( x^{\prime} + \beta c t^{\prime} \right)\\
                  & y = y^{\prime} \\
                  & z = z^{\prime} \\ 
                  & t = \gamma \left(t^{\prime} + \frac{\beta x^{\prime}}{c}\right)
                \end{cases}       
            \end{equation}
            and because we set $x^{\prime} = 0$, from Equation \ref{eq: lorentz_transfer_general}, 
            the two events of two successive photons collide with the electron can be 
            expressed as:
            $\left(\gamma v t_{1}^{\prime}, 0, 0, \gamma t_{1}^{\prime}\right)$ and 
            $\left(\gamma v t_{2}^{\prime}, 0, 0, \gamma t_{2}^{\prime}\right)$. 
            As Figure \ref{fig: inverse_compton_time_interval} shows, the time interval of two 
            successive photons (reciprocal of frequency) in frame $S$ is: 
            \begin{eqnarray}
              \label{eq: inverse_compton_time_interval}
              \Delta t &=& \left(t_2 - t_1\right) + \frac{\left(x_2 - x_1\right) \cos{\theta}}{c}  \nonumber \\
                      &=& \gamma \left(t_{2}^{\prime} - t_{1}^{\prime}\right) + \frac{\gamma v \left(t_{2}^{\prime} - t_{1}^{\prime}\right) \cos{\theta}}{c} \nonumber \\
                      &=&  \Delta t^{\prime} \gamma \left(1 + \beta \cos{\theta}\right) 
            \end{eqnarray}
            Equation \ref{eq: inverse_compton_time_interval} shows that the relationship of photon 
            number density between frame $S$ and $S^{\prime}$ is:
            \begin{equation}
              \label{eq: inverse_compton_number_density_relationship}
              n^{\prime} = n \gamma \left(1 + \beta \cos{\theta}\right) 
            \end{equation}
            Combine Equations \ref{eq: inverse_compton_number_density_relationship} and 
            \ref{eq: doppler_shift} we can transfer the incident photon energy density from 
            frame $S$ to $S^{\prime}$ according to Equation \ref{eq: inverse_compton_energy_density}
            \begin{equation}
              \label{eq: inverse_compton_energy_density}
              U_{rad}^{\prime} = U_{rad} \left[\gamma \left(1 + \beta \cos{\theta}\right)\right]^{2}
            \end{equation}
            In Equation \ref{eq: inverse_compton_energy_density}, the incoming photon energy density 
            is a function of the incident angle ($\theta$), in order to get the total photon 
            energy density, we integrate the equation over $\theta$. Then we get: 
            \begin{equation}
              \label{eq: inverse_compton_energy_density_total}
              U_{rad}^{\prime} = \frac{4}{3} U_{rad} \left(\gamma^2 - \frac{1}{4}\right)
            \end{equation}
            Combine Equations \ref{eq: inverse_compton_energy_density_total} and 
            \ref{eq: final_relationship}, the total scattered radiation power is:
            \begin{eqnarray}
              \label{eq: inverse_compton_power}
              P^{\prime} &=& P  \nonumber \\
                        &=& \frac{4}{3} \sigma_{T} c U_{rad} \left(\gamma^2 - \frac{1}{4}\right)
            \end{eqnarray}
            As mentioned before, $P^{\prime}$ and $P$ are the total radiation powers after 
            scattering. Before the low energy photons gain energy, they give away some energy 
            first which is $\sigma_{T} c U_{rad}$. 
            Therefore, we have to subtract this value from Equation \ref{eq: inverse_compton_power} 
            to calculate the rate of net energy gain, which is described by Equation 
            \ref{eq: inverse_compton_net_gain}.
            \begin{eqnarray}
              \label{eq: inverse_compton_net_gain}
              P^{\prime} = P = \frac{dE}{dt} &=& \frac{4}{3} \sigma_{T} c U_{rad} \left(\gamma^2 - \frac{1}{4}\right) - \sigma_{T} c U_{rad} \nonumber \\
                                            &=& \frac{4}{3} \sigma_{T} c U_{rad} \beta^{2} \gamma^{2}
            \end{eqnarray}
            If we compare Equation \ref{eq: inverse_compton_net_gain} with 
            Equation \ref{func: sync_combine}, we find that the form is very similar between these 
            two equations. 
            \begin{equation}
              \label{eq: comparision_inverse_compton_and_sync}
              \frac{P_{IC}}{P_{sync}} = \frac{U_{rad}}{U_{B}}
            \end{equation}
            where $U_{B}$ is the energy density of environment magnetic field. 

            \vspace{1cm}
            \singleFig{inverse_compton_time_interval}{0.45}{Two photons collide with an 
              electron. In the frame $S^{\prime}$, two photons collide with a rest electron 
              successively. In the frame $S$, the electron is no longer at rest and the 
              positions of the two events are $x_1$ and $x_2$}
          % \add{spectrum property, shape}
        
        \subsection{Curvature Radiation}
          Curvature radiation is the main source of gamma-ray photons. Charged particles  
          not only move around magnetic field lines (synchrotron radiation), 
          but also along magnetic field lines (curvature radiation) because the magnetic 
          field is very strong.

          Equation \ref{func: sync_total_power_precise} shows that total synchrotron radiation 
          power is related to pitch angel (because of the term $B_\perp$), and if the 
          pitch angel is $0$, there will be no synchrotron radiation. However, curvature 
          radiation can still be generated and is dependent on the curvature radii of 
          magnetic field lines:
          \begin{equation}
            \label{func: curvature_radiation}
            P = \frac{2}{3}\frac{e^2c}{s^2}\gamma^4
          \end{equation}
          where $s$ is the curvature radii of magnetic field lines. 
          Equation \ref{func: curvature_radiation} is very similar to Equation 
          \ref{func: sync_total_power_simplified} and the only difference is that 
          $R$ is changed to $s$. Similarly, according to the Equation 
          \ref{func: syncrothron spectrum}, the power spectrum of curvature radiation can 
          be written as:
          \begin{equation}
            \label{func: curvature spectrum}
            P = \frac{\sqrt{3}e^2}{m_es}\gamma \left(\frac{\nu}{\nu_c}\right) \int_{\nu / \nu_c}^{\infty} K_{5/3}\left(\eta \right)d\eta
          \end{equation}
          where $\nu_c$ is the critical frequencies of curvature photons and equals to:
          \begin{equation}
            \label{func: curvature critical frequency}
            \nu_c = \frac{3}{4\pi}\frac{c}{s}\gamma^3
          \end{equation}
          According to Equations \ref{func: curvature_radiation}, \ref{func: curvature spectrum} 
          and \ref{func: curvature critical frequency}, the spectral properties of curvature 
          radiation are similar to synchrotron radiation. The only differences in the 
          equations are particles' Lorentz factors and curvature radii. 

      \section{Millisecond Pulsar} 
        \subsection{$\mbox{P}$-$\dot{\mbox{P}}$ Diagram} 
          $P$-$\dot{\mbox{P}}$ diagram is an important tool for analyzing evolutions of 
          pulsars. Period ($\mbox{P}$) and time derivative of period ($\dot{\mbox{P}}$) are 
          two of pulsars' important characteristics. Analyzing the position of a pulsar in 
          the $P$-$\dot{\mbox{P}}$ diagram can give us some valuable information such as 
          what evolution stage the pulsar is in or the type of the pulsar, etc. 
          Figure \ref{fig:p-pdot} is an example of a $P$-$\dot{\mbox{P}}$ diagram. 
          The horizontal axis is pulsars' rotation periods and the vertical axis is first time 
          derivative of rotation periods ($\dot{\mbox{P}}$).
          \begin{figure}[h]
              \centering
              \includegraphics[scale=0.55]{/Users/grewwc/Desktop/Thesis/ppdot.png}
              \caption{\protect Position of pulsars in $\mbox{P}$-$\dot{\mbox{P}}$ diagram}
              \label{fig:p-pdot}
          \end{figure}
          In this $P$-$\dot{\mbox{P}}$ diagram, the negative slope lines represent the strengths 
          of surface magnetic fields while the positive slope lines represent the characteristic 
          ages of pulsars. The following is a short explanation for this. 
          From previous discussion, we have known that the 
          characteristic age of a pulsar is $\tau=-P/\dot{P}=P/(-\dot{P})$, so line of constant 
          $\tau$ is a set of straight lines with equal positive slope. We also know 
          $B\propto\sqrt{P\dot{P}}$, therefore the line of constant $B$ should be a part 
          of hyperbola. However, a $\mbox{P}$-$\dot{\mbox{P}}$ diagram is usually in log-log 
          scale. Hence, if we let $P^{\prime}=\log{P}$ and $\dot{P}^{\prime}=\log{\dot{P}}$,
          $B\propto\sqrt{P\dot{P}}$ turns to 
          $B\propto 1/2 \left(P^{\prime} + \dot{P}^{\prime}\right)$, 
          which represents a straight line.

          Figure \ref{fig:p-pdot} shows that most pulsars lie in the position of about 
          $1$s, $10^{-14}$s/s. At the same time, a couple of stars lie at the bottom-left 
          of Figure \ref{fig:p-pdot} --- these are millisecond pulsars (MSPs). Their 
          rotation periods are about 1-20 milliseconds. It is believed that MSPs are spun 
          up by accretion of mass from their companion stars. In the above 
          $P$-$\dot{\mbox{P}}$ diagram, we can observe that millisecond pulsars' 
          surface magnetic fields are about three to four orders of magnitude lower than those
          of normal pulsars. However, an MSP has a relative strong magnetic field near its 
          light cylinder ($B_{lc}$). For instance, PSR J1939+2134's $B_{lc}$ is about 
          $1.02\times10^5$ which is larger than Crab pulsar ($9.55\times10^5$) \cite{ATNF}.
          The reason is that an MSP's radius 
          of light cylinder ($R_{lc}=c/\omega)$ is much smaller than a normal pulsar's 
          because of its short rotation period and the magnetic field near light cylinder can
          be estimated as $B_{lc}\sim\left(R/R_{lc}\right)^3$. At the same time, 
          pulsars' emission mechanism is closely related to their magnetic field near the light 
          cylinder. As a result, like a normal pulsar, an MSP also can have broadband spectrum 
          from radio to gamma rays. 
          \subsection{Origin Of Millisecond Pulsars}
            From pulsars' emission mechanisms, we know that magnetic field of a pulsar 
            decreases with time while the spin period increase with time. But MSPs' spin 
            periods are much shorter than normal pulsars and surface magnetic fields are a 
            lot weaker. This makes an MSP seem to be both young and old. As a result,
            people think millisecond pulsars are old pulsars spun up by their companions. 
            The companion stars transfer mass and angular momentum to accelerate the rotation
            speed of pulsar. Therefore, the aged pulsar can spin faster gradually. 
            \subsubsection{Mass Transfer And Accretion In Binary Systems}
              X-ray binaries are a type of binary systems that is luminous in X-ray band. 
              There are several kinds of X-ray binaries including low mass X-ray binaries 
              (LMXB) and high mass X-ray binaries (HMXB). The mechanism of transferring mass 
              is different in these two types of systems. Before discussing mass 
              transfer, I will introduce Roche Lobe briefly. Figure 
              \ref{fig:roche lobe} is a schematic diagram of Roche Lobe.
              \begin{figure}[h]
                \centering
                \includegraphics[scale=0.5]{/Users/grewwc/Desktop/Thesis/roche_lobes.jpg}
                \caption{A schematic diagram of Roche lobe.\protect $L_{1}$ is called inner 
                          Lagrange point which is the intersection of equipotential lines 
                          of star A and B.}
                \label{fig:roche lobe}
              \end{figure}
              We call two stars in an LMXB as A and B respectively for convenience. It is 
              obvious that if an object is close to star A, the gravitational influence of A 
              is so strong that the gravitational effect of the star B can be ignored. Similarly, 
              this is true for star B. As a result, there must be a point where the effect 
              of star A is equal to star B which is called inner Lagrange point 
              \cite{0004-637X-603-1-283}. The two parts inside the largest equipotential 
              lines of A and B are called Roche lobe. If star B crosses
              its Roche lobe, than its mass will be attracted by A thus mass transfer between 
              A and B happens. This is the main way of mass transfer in 
              LMXB. While in HMXB, the mass can be transferred by strong wind of the massive 
              companion star. 

              The process of mass transfer can change the distance between two 
              companion stars. If a low-mass star transfer mass to a high-mass companion star,
              the orbital separation will become larger. This can actually stop mass transfer 
              and is a negative feedback. On the contrary, mass transfer from high-mass star 
              to low-mass star will shrink the orbital distance.
                       
        \subsection{Class II MSPs}
          \begin{figure}[h]   
            \centering
            \includegraphics[width=6.9cm,height=7.6cm]{/Users/grewwc/Desktop/Thesis/bands.png}
            \caption{Pulse profiles of PSR B1937+21 in radio, X-ray and gamma-ray.
              \protect \cite{0004-637X-787-2-167}}
            \label{fig:class }
          \end{figure}	 
          Radio emissions are usually considered to be emitted above the polar cap, which means
          radio emissions and gamma-ray emissions are from different location of a pulsar's 
          magnetosphere. However, there are about 10 sources showing aligned pulse profiles in 
          radio and gamma-ray implying that radio emission may produced from outer 
          magnetosphere and they are called Class II MSPs \cite{0004-637X-744-1-33}.
          These pulsars have strong magnetic fields near the light cylinders. 
          Figure \ref{fig:class } is an example of aligned pulse profile.

      \section{Previous Work and Objectives}
        The three energetic millisecond pulsars PSRs J0218+4232, \\ 
        B1821-24 and B1937+21 show 
        broadband spectra across radio, X-ray and gamma-ray bands like young pulsars. Table 
        \ref{table: basic_information_3_msps} lists some basic information of the three MSPs. 
        The three MSPs are some of the most energetic MSPs as the table shows. Some researches
        of gamma-ray spectra properties have been done using \textit{Fermi Lat}. The spectra were 
        fitted with the power-law-with-exponential-cutoff model (PLExpCutoff) and the results 
        are shown in Table \ref{table: previous_spectra_property}. 

        \begin{table}[!htp]
          \centering
          \scalebox{0.8}{
          \begin{tabular}{|cccc|} 
            \hline 
            & J0218+4232 & B1821-24 & B1937+21 \\
            \hline 
            \hline 
            Period ($ms$) & $2.32$  & $1.56$ & $3.05$ \\
            Period Derivative ($\dot{P}$, $10^{-20} $ ms $\mbox{ms}^{-1}$) & $7.74$ & $10.5$  & $162$ \\             
            Spin Down Age ($10^{8} $yr) & $4.76$ & $29.9$ & $2.35$ \\ 
            Surface Magnetic Field ($10^{8}$ gauss) & $4.29$ & $22.5$  & $4.09$ \\
            Light Cylinder Magnetic Field ($10^{5}$ gauss) & $3.21$ &$7.40$ & $10.2$ \\
            Spin Down Energy ($\dot{E}$, $10^{35}$ $erg$ $s^{-1}$) & $2.4$ & $2.2$ & $11.6$\\ 
            \hline   
          \end{tabular}}
          \vspace{0.5cm}
            \centering
            \mycaption{A few intrisic properties of PSRs J0218+4232, B1821-24 and B1937+21. 
              The data is from ATNF Pulsar Catalogue. \cite{ATNF}}
            \label{table: basic_information_3_msps}
        \end{table}

        \begin{table}[!htp]
          \centering
          \scalebox{0.8}{
          \begin{tabular}{|cccc|} 
            \hline 
            & J0218+4232 & B1821-24 & B1937+21 \\
            \hline 
            \hline 
            Energy Range (GeV) & $0.1 \sim 100 $ & $0.1 \sim 100 $  & $0.5 \sim 6$ \\
            Photon Index $\Gamma$ & $2.0\pm0.1$ & $1.6\pm0.3$ & $1.43\pm0.87$   \\
            Cutoff Energy (Gev) & $4.6\pm1.2$ & $3.3\pm1.5$ & $1.15\pm0.74$\\
            Photon Flux ($10^{-8}$ $cm^{-2}$ $s^{-1}$)& $7.7\pm0.7$& $1.5\pm0.6$ & $1.22\pm0.23$ \\
            Energy Flux ($10^{-11}$ erg $cm^{-2}$ $s^{-1}$) & $4.56\pm0.24$ & $1.3\pm0.2$ & $1.98\pm0.32$ \\ 
            \hline   
          \end{tabular}}
          \vspace{0.5cm}
            \centering
            \mycaption{Spectra properties of the three MSPs from previous studies. 
              \cite{ATNF} \cite{0004-637X-787-2-167} \cite{J1939_old}}
            \label{table: previous_spectra_property}
        \end{table}
        In the paper \cite{J1939_old}, they only study the PSR B1937+21 with energy range from 
        $0.1$ GeV to $6$ GeV and another paper \cite{0004-637X-787-2-167} studied the MSP more 
        thoroughly with a wider energy range. The paper proposed that a power-law model is 
        preferred and the photon index is $2.38\pm0.07$ with a TS value of $112$. 
        
        These papers are a little bit old and now we have about 4-year more \textit{LAT} data. 
        Besides, in year 2015, Fermi team released Pass 8 data including many improvements, 
        such as better energy measurement and significantly improved effective area. 
        As a result, it is reasonable to redo the gamma-ray analysis with the newer dataset and 
        more observation data in order to gain more precise results. 

        Therefore, my main objective is to use the new data to redo the gamma-ray
        analysis of the three energetic MSPs mentioned above. Then, I will do a numerical
        simulation based on a theoretical model called two-lay model and test if the the 
        predictions of the model are consistent with the new observation data. And finally, based 
        on numerical simulations of the two-layer emission model, I will generate broadband 
        spectra (including hard X-ray band and gamma-ray band) for all the three MSPs. 

    
    \chapter{Gamma-Ray Analysis}
      As mentioned before, because of very short rotation periods, MSPs usually have very small 
      light cylinders compared with normal pulsars. As a result, their emission mechanisms are 
      similar to normal pulsars, especially for my target objects --- PSRs J0218+4232, B1937+21 
      and B1821-24, which are among the fastest spinning and most energetic MSPs. Therefore, 
      as normal pulsars, the three MSPs have broadband emissions so it is intriguing to analyze 
      the spectra properties of them in gamma-ray band.
    
      \section{Introduction to the \textit{Fermi Gamma-ray Space Telescope}}
        The \textit{Fermi Gamma-ray Space Telescope} was launched on June 11, 2008 and opened 
        a new window of studying supermassive black-hole systems, pulsars and so on. Its 
        original name was \textit{Gamma-ray Large Area Space Telescope} (GLAST) and changed 
        to \textit{Fermi Gamma-ray Space Telescope} in honor of the great scientist Enrico Fermi. 

        The \textit{Fermi Gamma-ray Space Telescope} contains two parts: 
        \textit{Gamma-ray Burst Monitor (GBM)} and \textit{Large Area Telescope (LAT)} and the 
        latter is the main instrument which is at least 30 times more sensitive than 
        all gamma-ray telescopes launched before. I only use LAT for my purposes. Thus I 
        focus on the LAT instrument, which contains four main subcomponents including trackers, 
        calorimeters, anti-coincidence detectors and data acquisition systems. The reason why the 
        telescope is designed in this way is that high-energy gamma-rays cannot be refracted by
        lens or mirrors. Therefore, the working principals of the \textit{Fermi Lat} and other 
        gamma-ray telescopes are completely different.

      \begin{figure}[!htp]  
        \centering
        \includegraphics[scale=0.7]
                {/Users/grewwc/Desktop/Thesis/Gamma_telescope_schematic.png}
        \caption{The figure (\blackhref{https://www-glast.stanford.edu/instrument.html}
                {https://www-glast.stanford.edu/instrument.html})
                illustrates how \textit{Fermi Lat} tracks incident gamma-ray photons.}
          \label{fig:fermi schematic}
      \end{figure}
      Figure \ref{fig:fermi schematic} demonstrates the very basic idea of the 
      \textit{Fermi Lat} working principles. 

      \begin{itemize}
        \item Gamma-ray photons can enter the anti-coincidence detector freely while 
          cosmic-rays will generate signals which then tell the data acquisition system 
          component to reject these particles. In this way, the \textit{Fermi Lat} can 
          distinguish the gamma-ray photons from high energy cosmic-rays with a confidence 
          level over 99.9\%.
        \item The conversion foil (Figure \ref{fig:fermi schematic}) can convert  
          gamma-ray photons into electron and positron pairs. This procedure makes it 
          possible to determine the directions of the incident gamma-ray photons. 
        \item The tracker (particle tracking detectors in Figure \ref{fig:fermi schematic}) 
          records the positions of the electrons and positrons generated from the gamma-ray 
          photons. There are many trackers so the paths of particles can be constructed by 
          numerical simulations.
        \item When the electrons and positrons reach the calorimeter, their energies can be
          measured. By calculating the energy relationships between the gamma-ray photons and 
          the corresponding electrons and positrons, the energies of the original gamma-ray 
          photons can also be obtained. 
        \item The data acquisition system is like a filter of gamma-ray photons which can 
          reject unwanted particles such as cosmic-rays. Also, photons come from the Earth's 
          atmosphere are also rejected. 
      \end{itemize}

      For a telescope, the ability of measuring the directions and energies of incident 
      photons is very crucial. From the above descriptions of the \textit{Fermi Lat} 
      working principles, we know that the precision of construction of particles' paths 
      heavily influences how well we can measure the directions and energies of gamma-ray 
      photons. And this process is greatly dependent on simulations and available Fermi datasets,
      which means that with the improvements of simulations and datasets,
      the precision and sensitivity of \textit{Fermi Lat} can also be improved. The Pass 8 data 
      have reprocessed the entire Fermi mission dataset so the quality of the dataset is much 
      better. This is the main reason why I redo the analysis of the three MSPs.

      \section{A Brief Introduction to Fermi Data Analysis}

        \subsubsection{Processes of Doing Fermi Analysis}
          When doing \textit{Fermi Lat} data analysis, I basically dealt with two parts. The 
          first part is processing observation data and the second is generating photon 
          distributions based on spectra models. Cleaning data is straightforward including data 
          selection, data filtering with good time intervals (GTIs), generating count maps 
          and so on. Generating model-based count maps and count cubes needs a little bit 
          more efforts and mainly includes the following procedures. 

          Firstly, I need to generate a spectra model of all sources in our region of 
          interest (ROI). The model basically describes how strong each source is in 
          different energy bands and different positions. The initial parameters of the 
          model are from Fermi database. I do not fit positions of both point sources and 
          diffuse sources when do the data analysis. 
          However, the model alone is not very 
          helpful and I have to know other informations in order to simulate photon 
          distributions discussed as the following.

          Since I am going to compare my simulation with the observation data, I have to take 
          the telescope states into account. For example, the effective area of telescope 
          decreases when away from the optical axis. In addition, inclination angles and 
          observation time intervals have direct influences on the number of photon counts. 
          In short, after I get the simulated photon distribution from a model, it is 
          necessary to transfer the initial simulation into the real simulation by applying 
          the telescope functions. 
          
          After obtaining the photon distributions and spectra simulations, I then can do 
          comparisons in order to get the maximum likelihood of the model. I can divide the 
          total energy band into many smaller energy bins and denote
          the number of photon counts in observation data as $n_{i}$, so that 
          $\sum_{i}^{}n_{i} = N$, where $N$ is the total number of photons in full energy range. 
          The observed number of photon counts in $ith$ bin is a Poisson distribution with a 
          mean value of $m_{i}$. In fact, the value $m_{i}$ is the expected number of photon counts 
          from our spectra model. Therefore, the distribution for $ith$
          bin can be expressed by Equation \ref{func: maximum_likelihood_poisson}, where 
          $P_{i}\left(n_{i}\right)$ is the possibility of observing the $n_{i}$ photon counts 
          for the $ith$ bin. 

          \begin{equation}
            P_{i}\left(n_{i}\right) = \frac{e^{-m_{i}} m_{i}^{n_{i}}}{n_{i}!}
            \label{func: maximum_likelihood_poisson}
          \end{equation}

          As a result, it is not hard to generalize the possibility from each bin to all bins, 
          just by multiplying the possibilities for different bins.
          \begin{eqnarray}
            P_{total} &=& \prod_{i}^{}P_{i}\left(n_{i}\right) \nonumber \\ 
                      &=& e^{-\sum_{i}^{}m_i}\prod_{i}^{}\frac{m_{i}^{n_i}}{n_i!}
            \label{func: maximum_likelihood_poisson_all}
          \end{eqnarray}
          In Equation \ref{func: maximum_likelihood_poisson_all}, $n_i$ is directly 
          from observation data so they usually can not be changed during the binned likelihood 
          analysis. However, by changing the model, the $m_i$ can be altered. 
          Hence, my aim is to tweak the spectra model in order to make the total possibility 
          $P_{total}$ as large as possible. 

          This is the basic idea and procedure of doing Fermi data analysis. After doing these,
          I can go further such as testing how significant the targets are by creating TS 
          maps. The thesis basically follows the procedures. 

          Before finishing this part, I should briefly introduce the basic idea of TS maps. 
          TS value stands for Test Statistic value which can be expressed as Equation 
          \ref{func: ts_definition} 
          \begin{equation}
            TS = -2 \ln{\frac{L_{max,0}}{L_{max,1}}}
            \label{func: ts_definition}
          \end{equation}
          where $L_{max,0}$ and $L_{max,1}$ are the maximum likelihood of models in which the 
          target source is not included and included respectively. According to Equation 
          \ref{func: ts_definition}, the larger the TS value is, the larger $L_{max, 1}$ is, 
          which means that the probability of existence of the target source is larger. 
          In order to generate a TS map, I divide the whole map into many sub-grids. In each 
          sub-grid, the ``gtlike'' algorithm basically does two things. The first procedure is 
          calculating the maximum likelihood value directly based on the spectra model 
          ($L_{max,0}$). Then it adds an imaginary point source in the sub-grid, fits the source 
          and gets the maximum likelihood ($L_{max, 1}$) value. Therefore, I get two 
          maximum likelihood values. In the end, it calculate the TS value for the 
          sub-grid according to Equation \ref{func: ts_definition}. 

          After calculating the TS values for all sub-grids, I can generate a TS map just by 
          rendering colors according to each grid's TS value. By comparing TS values of 
          all sub-grids in a TS map, I can determine where the target source is most likely to 
          be and how large the probability is. 
          
          Generally speaking, for each source, I generate two TS maps and determine how likely 
          my target source is observed. For instance, if the data show the source is observed, 
          then the value of each pixel of the TS map containing the source should be low. 
          On the contrary, the TS values of the pixels around the position of the target 
          source should be significantly higher than other positions in the TS map if the target 
          source is deleted from the fitted spectra model.
       
        \section{\textit{Fermi Lat} Data Analysis}
          The basic idea of fitting spectra parameters is to make the count cube generated by 
          the model be as similar to the observation data as possible. A count cube is just a 
          combination of many count maps in different energy bands. For example, a dataset whose 
          energy range is from $100\mbox{MeV}$ to $100\mbox{GeV}$ can be divided into 30 bins. 
          I can generate a count map in each energy bin, thus I have 30 count maps.
          
          A count map is basically generated by the following steps. Firstly, choose a 
          pixel with a certain size. Then check each photon's direction to determine if 
          the photon is in this pixel. If it is in the pixel, the photon counts of the 
          pixel will add one. Therefore the more photons fall within the pixel, the more 
          photon counts the pixel has, hence the brighter the pixel is. By doing the same 
          thing for every pixel in the ROI, a count map can be generated. A count map can show 
          what has been observed intuitively and offers a very basic idea of if 
          the desired data is processed rightly.
          
          The calculation process can be summarized as follows. First of all, I have to generate 
          a spectral model for every known source in the region of interest (ROI) based on the Fermi 
          database. The database includes \textit{LAT} four-year Point Source Catalog (3FGL), 
          Galactic diffuse emission (gll\_iem\_v06.fits) and the isotropic emission 
          (iso\_P8R2\_SOURCE\_V6\_v06.txt). Then I can produce a count cube based on the 
          model. Generally speaking, the differences of the count cubes between the model and 
          observation is obvious. Then, the Fermi software adjusts the parameters 
          to make the difference smaller. Until the errors are acceptable, the software 
          outputs the final fitted parameters of corresponding spectral models.    

          I use a power-law with exponential-cutoff (PLExpCutoff) model to fit the 
          observation data and it is a special case of the power-law with 
          super-exponential-cutoff (PLSuperExpCutoff) model. The spectra of PLSuperExpCutoff 
          can be described by Equation \ref{eq: fermi_model}:  
          \begin{equation} 
            \label{eq: fermi_model}
            \frac{dN}{dE} = N_{0} \left(\frac{E}{E_0}\right)^{\gamma_1}\mbox{exp}\left[-\left(\frac{E}{E_c}\right)^{\gamma_2}\right]
          \end{equation}  
          where $N_0$ is called prefactor, $E_c$ is the cutoff energy and the $E_0$ is a scale 
          parameter. PLExpCutoff model is the special case where $\gamma_2=1$. My aim is to 
          fit the parameters $N_0$, $E_c$ and $\gamma_1$ to make the model be more 
          consistent with the \textit{Fermi Lat} observation data.

        \subsection{Verifying the Data Analysis Process}
          Before analyzing the observations of my target sources, it is reasonable to test 
          if my procedures of data processing are right. In order to do so, I try to do 
          analysis for two bright pulsars PSRs J0007+7303 and J0534+2200. The reason I choose 
          these two pulsars is that according to previous studies, they are bright and easy 
          to detect with a large TS value of 43388 and 102653 for J0007+7303 and J0534+2200 
          respectively. \cite{0067-0049-208-2-17} 

          In the spectra fit process, I do not use the same fit parameters as the previous 
          paper (for instance, the number of free parameters in the ROI is different), 
          however, I get similar results in terms of spectra index.  
          In Table \ref{table: previous_result_comparison}, I used the observation data from 
          2009-01-01 to 2013-02-01 in order to try to be consistent with the old paper 
          \cite{0067-0049-208-2-17}. In addition, I also fit spectra with observation data up 
          to 2018-02-01 and \textit{Pass 8 dataset} to test how big improvement I can make with 
          the new \textit{Fermi Pass 8} dataset and more observation data. 
          The results of year 2018 data are shown in Table \ref{table: 2018_fit_data}.

          Tables \ref{table: previous_result_comparison} and \ref{table: 2018_fit_data} mainly 
          show two pieces of information. Firstly, my procedures of dealing with observation 
          data has no obvious problems, so basically I can trust fit results of my target 
          sources. Secondly, the \textit{Fermi Pass 8 Lat Data} has improved the accuracy a 
          lot. For example, as Table \ref{table: previous_result_comparison} shows, the photon 
          indexes are $1.30\pm0.02$ and $1.4\pm0.1$, which shows that the errors reduce a lot. 
          Additionally, the TS value is more than double as before.
          \question{However, the cutoff energies are not consistent between the previous result and the new result.
            (I need to explain this a little bit later).}
          \vspace{1cm} 
          \begin{table}[!ht]
            \centering
            \scalebox{0.8}{
            \begin{tabular}{|c|c|c|c|c|c|c|} 
              \hline 
              & \multicolumn{3}{|c|}{Test Results} & \multicolumn{3}{|c|}{Previous Results} \\ 
              \cline{2-7}
              & $\Gamma$ & $E_c$ (MeV) & TS & $\Gamma$ & $E_c$ (MeV) & TS \\ 
              \hline
              J0007+7303 & $1.30\pm0.02$ & $2010\pm85$ & $96979$ & $1.4\pm0.1$ & $4700\pm200$ & $43388$  \\
              \hline 
              J0534+2200 & $2.07\pm0.01$ & $9880\pm572$ & $239015$ & $1.9\pm0.1$ & $4200\pm200$ & $102653$  \\
              \hline
            \end{tabular}}
            \vspace{0.5cm}
              \centering
              \mycaption{The spectra fit results with data from 2008 August 4 to 2011 August 4. 
                In the thesis, in order to make data 
                analysis more convinient, I use some pipeline scripts to deal with the 
                observation data. The "Test Results" column is the results generated by 
                using the pipeline scripts. The "Previous Results" column lists the 
                corresponding spectra properties based on the previous studies 
                \cite{0067-0049-208-2-17}. According to the standard PLExpCutoff model 
                (described in equation \ref{eq: fermi_model}), $\Gamma$ is photon index 
                and $E_c$ is cutoff energy.}
              \label{table: previous_result_comparison}
          \end{table}
          \vspace{1cm}            

          \begin{table}[!ht]
            \centering
            \scalebox{0.85}{
            \begin{tabular}{|cccc|}
              \hline 
              &$\Gamma$& $E_c$ (GeV) & TS value\\ \hline \hline
              J0007+7303 & $1.34\pm0.02$ & $2.20\pm0.67$ & $210166$ \\  
              J0534+2200 & $2.01\pm0.01$ & $9.20\pm0.37$ & $449946$ \\
              \hline
            \end{tabular}}
            \mycaption{Fit results with data from year 2009 to year 2018. The physical 
              meanings of $\Gamma$ and $E_c$ are the same as Table 
              \ref{table: previous_result_comparison}}
            \label{table: 2018_fit_data}
          \end{table}
          % \vspace{1cm}            

        \section{PSR J0218+4232}
          % \label{j0218}
          The ROI is a circle with radius of $20^\circ$ and all parameters of sources which 
          are $8^\circ$ outside of the center are fixed. For sources within $8^\circ$, initial
          values of parameters are the same as their default values according to \fgl.
          In this case, there are seven point sources which have free parameters. In Figure
          \ref{fig: j0218_count_map_and_model}, the green circles represent those sources.
          There are some very bright sources which have no free parameters
          in the outer parts of the count map. The reason is that the they are so far away 
          from PSR J0218+4232 that \textit{Fermi Lat} can distinguish if a photon comes 
          from the target source or the outer sources. As a result, I do not need to fit 
          any parameters for those outer sources and their spectra properties are from 
          \fgl. However, it is different for the nearby sources and they have to be fitted.
          
          \subsection{Count Maps And Count Cubes}
            \begin{figure}[!ht]  
              \begin{center}
              \begin{minipage}{0.45\textwidth}
                \begin{center} 
                    \includegraphics[scale=0.33]{/Users/grewwc/Desktop/Thesis/j0218_count_map_with_region.png}
                \end{center}
              \end{minipage}
              \begin{minipage}{0.45\textwidth}
                \begin{center} 
                    \includegraphics[scale=0.33]
                        {/Users/grewwc/Desktop/Thesis/j0218_count_map_model.png}
                \end{center}
              \end{minipage}
            \end{center}
            \begin{center}
              \caption{The count map of PSR J0218+4232 (left) and the count map generated by 
              the model (right). In the left panel, the green circles represent sources needed 
              to be fitted. The right panel is a count map created according to the fitted 
              spectra model. The size of each figure is 141 pixels $\times$ 141 pixels, 
              and the dimension for each pixel is $0.2^\circ \times 0.2^\circ$.}
              \label{fig: j0218_count_map_and_model}  
            \end{center} 
          \end{figure}

          The left part of Figure \ref{fig: j0218_count_map_and_model} is the count map of 
          PSR J0218+4232 created directly from the observation data. In the center of the left 
          panel, the target source can be seen clearly.
          The dimensions of the figures seems to be weird and the reason why the count 
          map is $141$ pixels wide is that I need to select a circle region from the original 
          data. However, when generating a count map, I have to assign the sizes for x and y 
          axis separately, which means that the a count map is actually a rectangular. As a 
          consequence, I have to crop a rectangular from the original circle region and usually, 
          the rectangular is chosen as a square. 
          
          \begin{figure}[!htp]
            \begin{minipage}{0.32\textwidth}
              \begin{center} 
                \includegraphics[scale=0.28]
                      {/Users/grewwc/Desktop/Thesis/j0218_ccube_start.png}
              \end{center}
            \end{minipage}
            \begin{minipage}{0.32\textwidth}
              \begin{center}
                \includegraphics[scale=0.28]
                      {/Users/grewwc/Desktop/Thesis/j0218_ccube_middle.png}
              \end{center}
            \end{minipage}
            \begin{minipage}{0.32\textwidth}
              \begin{center}
              \includegraphics[scale=0.28]
                    {/Users/grewwc/Desktop/Thesis/j0218_ccube_end.png}
              \end{center}
            \end{minipage}
            \caption{Three count maps from PSR J0218+4232's count cube. The energy ranges of 
              the figures are 100$\sim$123MeV, 1.873$\sim$2.310GeV, 35.11$\sim$43.29GeV 
              respectively.}
            \label{fig: j0218_ccube_bin_1_and_15}
          \end{figure}
          
          Figure \ref{fig: j0218_ccube_bin_1_and_15} is a comparison between PSR J0218+4232's 
          count maps in different energy bands. The count map in about $100\mbox{MeV}$ is so 
          messy that we can hardly distinguish the target source while the energy is above 
          $30\mbox{GeV}$ there are so few photons that there is not a clear sign of the source. 
          I choose three circle regions whose centers are the target sources and the radii 
          are 1000 $''$ for all of the three figures and then calculate the total numbers of 
          photon counts of the selected regions. 
          As Table \ref{table:j0218_ccube_photon_counts} shows, though total number of photon 
          counts around the target source is similar between the left and middle count maps, 
          the numbers of counts per energy are much different. Since there are few photons 
          in high energy bands (above $50\mbox{GeV}$) compare to other energy bands, I focus 
          more on the lower energy part. 

          \begin{table}[!htp]   
            \centering
            \scalebox{0.85}{
            \begin{tabular}{|m{4.5cm}ccc|}
              \hline 
              & Left & Middle & Right \\
              \hline \hline 
              Total counts & 78 & 93 & 0 \\
              Energy range ($MeV$) & 100$\sim$123 & 1873$\sim$2310 & 35110$\sim$43290 \\ 
              Counts / MeV ($MeV^{-1}$)& 3.39 & 0.21 & 0.00 \\  
              \hline
            \end{tabular}}
            \mycaption{Numbers of photon counts of count maps in different energy bands for 
                  PSR J0218+4232.}
            \label{table:j0218_ccube_photon_counts}
          \end{table}
          \subsection{Binned Likelihood Analysis}
            Figure \ref{fig: j0218_count_map_and_model} shows that the fit results of the model 
            are consistent with the observation. However, there are lots of small red pixels 
            in the left panel (generated directly by the observation data) while the 
            right panel is very "clean". This means that a lot of photons are thought as 
            generated by the modeled source. Thus in the model, the sources are generally 
            slightly brighter than the observation. However, the target
            source is an exception. In the region I have used before (the center is the 
            target source, and the radius is 1000$''$), the total photon counts in the left 
            panel are 1815 compare to 1737 in the right panel. 

            The reason why the count map generated directly by the observation data is a lot 
            more messy is that the source model is generated based on the Fermi database 
            and their spatial position is fixed. This means that if 
            a photon comes from a particular direction and there is no any known pulsar in 
            that direction, this photon has to be classified into other directions and there 
            is a modeled source in the direction.  Thus, the spatial positions of photons are 
            different between the observation and the model, and the count maps generated 
            directly from models are usually cleaner. 
            
            % \begin{figure}[!ht]
            %   \begin{minipage}{1\textwidth}
            %     \begin{center} 
            %       \includegraphics[scale=0.6]{/Users/grewwc/Desktop/Thesis/j0218_count_map_diff.png}
            %     \end{center}
            %   \end{minipage}
            %   \centering
            %   \begin{minipage}{0.8\textwidth}
            %     \mycaption{The residual map shows the difference between the observation and the model. 
            %     It is generated by subtracting the photon counts of each pixel between the count maps of 
            %     observation and the model. 
            %     \change{ugly, scale may be wrong}}
            %     \label{fig: j0218_count_map_diff}
            %   \end{minipage}
            % \end{figure}

            \begin{figure}[!htp]
              \begin{center}
              \begin{minipage}{0.45\textwidth}
                \begin{center} 
                  \includegraphics[scale=0.4]
                        {/Users/grewwc/Desktop/Thesis/j0218_count_map_linear_scale.png}
                \end{center}
              \end{minipage}
              \begin{minipage}{0.45\textwidth}
                \begin{center}
                  \includegraphics[scale=0.4]
                        {/Users/grewwc/Desktop/Thesis/j0218_dif_map_linear_scale.png}
                \end{center}
              \end{minipage}
            \end{center}
            \caption{The count map and residual map of PSR J0218+4232.
              The figures are both in linear scale in order to compare the residual map 
              between the original count map more intuitively. The left panel is the count 
              map and the right panel is the residual map which shows the differences between 
              the observation and the spectral model. It is created by directly subtracting 
              the photon counts of each pixel between the count maps of observation data and 
              the spectral model. The green circle region represents (the regions are 
              completely the same in the two figures) the largest number photon counts of the 
              residual map and its radius is $2000''$.}
            \label{fig: j0218_count_map_diff}
            \end{figure}

            Figure \ref{fig: j0218_count_map_diff} basically describes how well the model is 
            compared to the observation data. There are some bright dots in 
            the residual map showing the differences between the spectral model and the 
            observation data. In the residual map of Figure \ref{fig: j0218_count_map_diff}, 
            most differences of absolute photon counts are small, however, in the green 
            circle region, the absolute value 6003 is large. This means that in this region,
            the number of photon counts of the observation data (21525) is 6003 larger than 
            in our model. This is not negligible since it is nearly $28\%$ of the original 
            photon counts. Does this mean that the model is not good? The answer should be 
            yes, however, this does not mean the fit is not good since the model parameters 
            in this region are all fixed and the fixed values are from the Fermi Lat 4-year 
            Point Source Catalog. Hence, the difference shows some problems of the spectral 
            model, but has nothing to do with the fit results. Instead, from the residual map, 
            we can see that the fit results are good because the differences
            of number of photon counts are very low, which are about $5\%$ of the photon 
            counts of the count map on average.

            Table \ref{table: j0218_fit_result} lists the results of the fit parameters. 
            We see from Table \ref{table: j0218_fit_result} that the new fit results are 
            consistent with the old ones. However, the precision improves a lot which is 
            ascribed to the \fgl{} and PASS 8 dataset. Figure \ref{fig: j0218_cur.png} is a 
            plot of the spectrum according to Function \ref{eq: fermi_model}.
            One thing should be noticed is that I need to multiply $E^2$ to Function 
            \ref{eq: fermi_model} 
            in order to get the flux. Figure \ref{fig: j0218_cur.png} shows that 
            the global fit is consistent with flux points fitted by each energy bin separately. 
            The TS value of PSR J0218+4232 is 7110, which gives 
            a significance level $\sigma \approx \sqrt{TS} \approx 84$. This strongly implies the 
            presence of the target source. I can also plot TS maps to test the presence of the 
            source as Figure \ref{fig: j0218_tsmap_comparison_20} shows. 
            
            \begin{table}[!htp]
              \centering
                \scalebox{0.8}{
                \begin{tabular}{|ccc|}
                  \hline
                  & This Study & Previous Results \\
                  \hline \hline 
                  Photon Index ($\Gamma$) & $1.89\pm0.04$ & $2.0\pm0.1$ \\
                  Cutoff ($E_c$, GeV) & $3.77\pm0.40$ & $4.6\pm1.2$ \\
                  Photon Flux ($10^{-8}$ $cm^{-2} s^{-1}$) & $7.29\pm0.28$ & $7.7\pm0.7$ \\
                  Energy Flux ($10^{-11}$ erg $cm^{-2} s^{-1}$) & $4.45\pm0.16$ & $4.56\pm0.24$ \\
                  TS value & $6809$ &  $1313$  \\
                  \hline
                \end{tabular}}  
                \mycaption{Fit parameters of the spectral model of PSR J0218+4232. 
                  The names of parameters are consistent with Equation
                  \ref{eq: fermi_model}. The previous results are from the paper \cite{0067-0049-208-2-17}.}
                \label{table: j0218_fit_result}        
            \end{table}  

            \singleFig{j0218_cur.png}{0.35}{The log-log plot of flux to energy of PSR J0218+4232. The grey shade represents 
              fitting errors, black points with error bars are flux points, the blue dots are upper values and the 
              red line is the PLExpCutoff model multiplied by $E^2$. Flux points 
              are fitted separately by dividing the total energy bin (100 MeV $\sim$ 100 GeV) into multiple energy bins.
              The horizontal error bars represents the width of each bin. }
            \vspace{1cm}
            % \singleFigureblank.png}{0.3}{\change{Figurereshould be a TS map, but the new TS map has not been generated yet.
            % The previous TS map does not use the best fit parameters, so I have to re-generate the TS map. And this 
            % world map is just a reminder and it may be used a lot.}}
            \begin{figure}[!htp]
              \begin{center}
              \begin{minipage}{0.46\textwidth}
                \begin{center} 
                  \includegraphics[scale=0.37]{/Users/grewwc/Desktop/Thesis/j0218_tsmap_with_source_20.png}
                \end{center}
              \end{minipage}
              \begin{minipage}{0.45\textwidth}
                \begin{center}
                  \includegraphics[scale=0.37]{/Users/grewwc/Desktop/Thesis/j0218_nosource_20.png}
                \end{center}
              \end{minipage}
            \end{center}
            \caption{TS maps of PSR J0218+4232. The figures' dimensions are 
            $4^{\circ} \times 4^{\circ}$ ($20$ pixels $\times$ $20$ pixels with 
            $0.2^{\circ} \times 0.2^{\circ}$ for each pixel). The left and the right panels are 
            generated by the xml models with and without the target source PSR J0218+4232 respectively.
            The left panel shows that the possibility of adding an imputative point source is very low 
            only with a maximum TS value of less than 5. However, the right panel strongly implies that 
            there should be an additional source after I have removed the target MSP from the spectral 
            model, which means it's highly likely that PSR J0218+4232 is contained in the observation 
            data.}
            \label{fig: j0218_tsmap_comparison_20}
            \end{figure}

            % \begin{figure}[!ht]
            %   \begin{center}
            %   \begin{minipage}{0.45\textwidth}
            %     \begin{center} 
            %       \includegraphics[scale=0.45]{/Users/grewwc/Desktop/Thesis/j0218_tsmap_with_source_20.png}
            %     \end{center}
            %   \end{minipage}
            %   \begin{minipage}{0.45\textwidth}
            %     \begin{center}
            %       \includegraphics[scale=0.45]{/Users/grewwc/Desktop/Thesis/j0218_nosource_20.png}
            %     \end{center}
            %   \end{minipage}
            % \end{center}
            %   \mycaption{TS maps for PSR J0218+4232. Figureres' dimensions are 
            %   $4^{\circ} \times4^{\circ}$ ($20$ pixels $\times$ 20 $pixels$ with 
            %   $0.25^{\circ} \times 0.25^{\circ}$ for each pixel). Figureres' meanings 
            %   are completely the same with Figure \ref{fig: j0218_tsmap_comparison_15}}
            %   \label{fig: j0218_tsmap_comparison_20}
            % \end{figure}
            % After obtaining the spectra fit results in gamma-ray band, we can generate a broad band spectrum. 
            % The hard X-ray data is from the paper  
            % \blackhref{https://arxiv.org/pdf/1704.02964.pdf}{NUSTAR HARD X-RAY OBSERVATIONS OF THE ENERGETIC   %
            % MILLISECOND PULSARS PSR B1821-24, PSR B1937+21, AND PSR J0218+4232}. We also generate simulation data
            % from the two-layer model. Then we compare the simulation and observation as the Figure
            % \ref{fig: J0218+4232} shown. The prediction of the two-layer model is consistent with the observation
            % both in X-ray band (from about 3 keV to 10 MeV) and high energy gamma-ray band (above 1 GeV). However, 
            % from about 100 MeV to 1 GeV, the spectrum from the two-layer model is not consistent with Fermi data. 
            % This can have 2 explanations. Firstly, the Fermi telescope is not sensitive in about 100 MeV.
            % As a result, the observation data may not be very reliable at about this energy band. Secondly, the  
            % real emission mechanism in the energy band is different from the model predicts. Thus, we can observe 
            % inconsistency between the simulation and observation.

            % \singleFigureJ0218+4232}{0.35}{\Notice{have to uniformly change the style of figures later.}}

            It is also instructive to check the count residuals of the fit as Figure 
            \ref{fig: j0218_count_spectra} shows. The galactic and isotropic emissions are very 
            bright compared with the target source and this can bring some difficulties to 
            the fit (especially for the other two MSPs PSRs B1937+21 and B1821-24). The residuals 
            and fluctuations become larger and more obvious when the energy is larger than 
            $10\mbox{GeV}$. At the same time, in low gamma-ray part 
            (from $100\mbox{MeV}$ to $200\mbox{MeV}$), the number of counts of the model 
            deviates from observation counts obviously because of the relatively low energy 
            resolution in low-energy gamma-ray part. The error of the first bin is larger
            than the next several bins in Figure \ref{fig: j0218_cur.png} also shows that 
            the fit in low-energy gamma-ray band (from $100\mbox{MeV}$ to $200\mbox{MeV}$) 
            is not as good as higher energy bands (but not too high). 
            
            \begin{figure}[!htp]
              \centering
              \includegraphics[scale=0.42]{/Users/grewwc/Desktop/Thesis/j0218_count_spectra.png}
              \caption{The count spectra and count residuals of PSR J0218+4232.
                      The upper panel is the count spectrum of all sources included in the 
                      fit procedure. Thick lines are those sources with free fit parameters 
                      while the thin lines are fixed sources. The green line and the black 
                      dots represents observed counts in different energy bands. The purple 
                      line represents galactic emission. The lower panel shows the count 
                      residuals in different energy bands. } 
              \label{fig: j0218_count_spectra}
            \end{figure}

            \section{PSR B1821-24}
              The ROI region is a also circle whose radius is $20^\circ$ and all 
              parameters of sources outside of $8^\circ$ are fixed. 
              There are six free sources in the region of $8^\circ$. Figure
              \ref{fig: b1821_count_map_with_region_and_model} 
              is a combination of count maps of observation data and the model. 

              \subsection{Count Maps And Count Cubes}
                \begin{figure}[!ht]
                  \begin{center}
                  \begin{minipage}{0.45\textwidth}
                    \centering 
                    \includegraphics[scale=0.27]{/Users/grewwc/Desktop/Thesis/b1821_count_map_with_region.png}
                  \end{minipage}
                  \begin{minipage}{0.45\textwidth}
                    \centering
                    \includegraphics[scale=0.27]{/Users/grewwc/Desktop/Thesis/b1821_count_map_model.png}
                  \end{minipage}
                \end{center}
                \mycaption{The count map of PSR B1821-24 (left) and the count map generated by 
                  the model (right). In the left panel, the green circles are free sources. The 
                  sizes of the both figures are 141 pixels $\times$ 141 pixels, and each 
                  pixel's dimension is $0.2^\circ \times 0.2^\circ$.}
                \label{fig: b1821_count_map_with_region_and_model}
              \end{figure}
            
              \begin{figure}[!ht]
                \begin{center}
                \begin{minipage}{0.31\textwidth}
                  \begin{center} 
                    \includegraphics[scale=0.27]{/Users/grewwc/Desktop/Thesis/b1821_ccube_start.png}
                  \end{center}
                \end{minipage}
                \begin{minipage}{0.31\textwidth}
                  \begin{center}
                    \includegraphics[scale=0.27]{/Users/grewwc/Desktop/Thesis/b1821_ccube_middle.png}
                  \end{center}
                \end{minipage}
                \begin{minipage}{0.31\textwidth}
                  \begin{center}
                  \includegraphics[scale=0.27]{/Users/grewwc/Desktop/Thesis/b1821_ccube_end.png}
                  \end{center}
                \end{minipage}
              \end{center}
              \mycaption{Three figures of PSR B1821-24's count cube. The energy ranges of the figures are  
                100$\sim$123MeV, 1.873$\sim$2.310GeV, 81.11$\sim$100GeV respectively from left to right.}
              \label{fig: b1821_ccube_1_15_33.png}
              \end{figure}

              The left and right panels of Figure \ref{fig: b1821_count_map_diff.png} are the count 
              map of the PSR B1821-24 generated from observation data and spectral model respectively. 
              Like the situations of PSR J0218+4232, the count map from the model is clearly cleaner than 
              from the observation data and the two figures are very similar, which implies that 
              the spectra model describes the observation data well.  

              Figure \ref{fig: b1821_ccube_1_15_33.png} are count maps of PSR B1821-24 in 
              different energy bands. The target pulsar is too faint in very high energy bands 
              and interfered too much by the ambient environment in low energy bands 
              (slightly above 100MeV). PSR B1821-24 is in the M28 globular cluster and is the most 
              energetic one, which is much brighter than other sources found in M28. However, since 
              PSR B1821-24 is very faint observed from Earth, it is understandable that the fit 
              results are not as good as the results of PSR J0218+4232. 
              \subsection{Binned Likelihood Analysis}
              \begin{figure}[!ht]
                \begin{center}
                \begin{minipage}{0.45\textwidth}
                  \begin{center} 
                    \includegraphics[scale=0.40]{/Users/grewwc/Desktop/Thesis/b1821_countmap_noregion.png}
                  \end{center}
                \end{minipage}
                \begin{minipage}{0.45\textwidth}
                  \begin{center}
                    \includegraphics[scale=0.40]{/Users/grewwc/Desktop/Thesis/b1821_count_map_diff.png}
                  \end{center}
                \end{minipage}
              \end{center}
              \caption{The count map and residual map of PSR B1821-24 in linear scale. 
                The reason why use linear scale is not used here is that the residual map is 
                nearly black in linear scale. The \textsf{left} figure is the count map and the 
                \textsf{right} figure is the residual map showing the difference between the 
                observation data and the spectral model.}
              \label{fig: b1821_count_map_diff.png}
            \end{figure}

            \singleFig{b1821_cur.png}{0.37}{The log-log plot of flux to energy of 
              PSR B1821-24's gamma-ray spectrum.}
            \vspace{1cm}

            The differences of the count map between the observation data and the model are described 
            as Figure \ref{fig: b1821_count_map_diff.png} which is in linear scale. Although there are 
            many red and blue dots in the right panel of the figure, their absolute values are 
            generally small compared with the original counts value.
            Thus, the fits are acceptable in general. But I am still trying to get the better spectra 
            model. 

            Table \ref{table: b1821_fit_result} shows the global fit results of PSR B1821-24. 
            The TS value of the model is 941 which gives us a significance level 
            of about $\sqrt{941} \sim 31$. This strongly supports the existence of the target source in 
            the observation data. As Table \ref{table: b1821_fit_result} shows, the energy flux from 
            100MeV to 100GeV is not consistent between the two studies and gamma-ray spectrum of the 
            previous study is also softer \cite{2013ApJ...778..106J}. 

            Figure \ref{fig: b1821_twolayer_cur.png} shows that the global fit is consistent with 
            the flux points generated by fitting sub-energy bins separately. We notice the upper value 
            for the first energy bin is slightly smaller than the global fit. 
            Though it is strange that the upper value is smaller than the normal value at first glance, 
            it is reasonable since the flux points are fitted separately and are independent to 
            the global fit. In fact, I use a single power-lay model to fit each sub-energy bin while 
            PLExpCutoff model to do the global fit. As discussed previously, the lower energy parts of 
            the observation (slightly above 100MeV) is not as reliable as other energy bands. As a result, the 
            separate fit for the first energy bin is not as good as the global fit and it is reasonable 
            that the two fit results are not completely consistent. When this happening, I have 
            more confidence on the global fit than the separate fit.
            
            Figure \ref{fig: b1821_tsmap_comparison_20} contains TS maps of PSR B1821-24. The 
            comparison of TS maps also show the significant of the target MSP. Figure 
            \ref{fig: b1821_count_spectra} shows how well the fit is. Like the count residuals of 
            the other two pulsars, the fit is not good in low energy part (from 100MeV to 500MeV)
            and high energy band (above 10GeV). Besides, as Figures \ref{fig: j0218_count_spectra},
            \ref{fig: b1821_count_spectra} and \ref{fig: j1939_count_spectra} show, the numbers of 
            observed photon events are all larger than modeled photon counts for PSRs J0218+4232, 
            B1821-24 and B1937+21. 

            \begin{table}[!ht]
              \centering
                \scalebox{0.8}{
                \begin{tabular}{|ccc|}
                  \hline 
                  & This Study & Previous Results \\
                  \hline \hline  
                  Photon Index ($\Gamma$) & $1.91\pm0.07$ & $1.6\pm0.3$ \\
                  Cutoff Energy($E_c$, GeV) & $4.50\pm0.71$ & $3.3\pm1.5$ \\
                  Photon Flux ($10^{-8}$ $cm^{-2} s^{-1}$) & $3.85\pm0.31$ & $1.5\pm0.6$ \\ 
                  Energy Flux ($10^{-11}$ erg $cm^{-2}$ $s^{-1}$) &$2.44\pm0.14$ & $1.3\pm0.2$\\
                  TS value & $941$ & $76$ \\
                  \hline
                \end{tabular}}  
                \mycaption{Fit parameters of the spectral model of PSR B1821-24. 
                  The names of parameters are also consistent with Equation
                  \ref{eq: fermi_model} The energy ranges of photon flux between the two results 
                  are different.}
                  
                \label{table: b1821_fit_result}        
            \end{table}  
            \vspace{1cm}
              % \singleFigureblank.png}{0.3}{\change{Figurereshould be a TS map, but the new TS map has not been generated yet.
              % The previous TS map does not use the best fit parameters, so I have to re-generate the TS map. And this 
              % world map is just a reminder and it may be used a lot.}}

            \begin{figure}[!ht]
              \begin{center}
              \begin{minipage}{0.46\textwidth}
                \begin{center} 
                  \includegraphics[scale=0.35]{/Users/grewwc/Desktop/Thesis/b1821_tsmap_with_source_20.png}
                \end{center}
              \end{minipage}
              \begin{minipage}{0.45\textwidth}
                \begin{center}
                  \includegraphics[scale=0.35]{/Users/grewwc/Desktop/Thesis/b1821_tsmap_nosource_20.png}
                \end{center}
              \end{minipage}
            \end{center}
              \caption{TS maps of PSR B1821-24. The figures' dimensions are $4^{\circ} \times4^{\circ}$ 
              ($20$ pixels $\times$ $20$ pixels with $0.2^{\circ} \times 0.2^{\circ}$ for each pixel). 
              The left and right panels are generated by the XML models with and without the 
              target source PSR B1821-24 respectively. The left panel shows that the possibility 
              of adding an imputative point source is very low only with a maximum TS value of 
              less than 11 while the TS values of the right figure are generally much larger.}
                \label{fig: b1821_tsmap_comparison_20}
            \end{figure}
            \vspace{1cm}
            
            % \add{lack something 
            %   1. TS values  
            %   2. count residuals}
            % \begin{figure}[!ht]
            %   \begin{center}
            %   \begin{minipage}{0.45\textwidth}
            %     \begin{center} 
            %       \includegraphics[scale=0.43]{/Users/grewwc/Desktop/Thesis/b1821_tsmap_with_source_20.png}
            %     \end{center}
            %   \end{minipage}
            %   \begin{minipage}{0.45\textwidth}
            %     \begin{center}
            %       \includegraphics[scale=0.37]{/Users/grewwc/Desktop/Thesis/b1821_tsmap_nosource_20.png}
            %     \end{center}
            %   \end{minipage}
            % \end{center}
            % \mycaption{TS maps of PSR B1821-24. Figureres' dimensions are $4^{\circ} \times4^{\circ}$ 
            % ($20$ pixels $\times$ $20$ $pixels$ with $0.25^{\circ} \times 0.25^{\circ}$ for each pixel). 
            % The meaning of Figure is the same as Figure \ref{fig: b1821_tsmap_comparison_15}}
            % \label{fig: b1821_tsmap_comparison_15}
            % \end{figure}
            % \vspace{1cm}
      
      
          \begin{figure}[!htp]
            \centering
            \includegraphics[scale=0.40]{/Users/grewwc/Desktop/Thesis/b1821_count_spectra.png}
            \caption{The count spectra and count residuals of PSR B1821-24.
                  The upper panel is the count spectrum of all sources included in the 
                    fit procedure. Thick lines are those sources with free fit parameters 
                    while the thin lines are fixed sources. The green line and the black 
                    dots represents observed counts in different energy bands. The purple 
                    line represents galactic emission. The lower panel shows the count 
                    residuals in different energy bands. } 
            \label{fig: b1821_count_spectra}
          \end{figure}


        \section{PSR B1937+21}
          \subsection{Phase Averaged Analysis}
            First of all, I use a PLExpCutoff model to fit the gamma-ray spectra. 
            In order to make the data analysis be more consistent, I choose the same parameters
            to process the raw observation data. Like the other two MSPs, the radius of the ROI 
            is $20^{\circ}$ degrees, and all parameters of sources $8^{\circ}$ degrees outside 
            from the center are fixed with default values. There are nine point sources including 
            the target source PSR B1937+21 and twenty-eight free parameters. In fact, PSR B1937+21
            is not included in \fgl{} may be because that the MSP is very weak and the signal to 
            noise ratio is so low that the reliable spectrum fit results with a large TS value 
            had not been obtained when the \fgl{} was published. Thus I have to add the 
            configuration file for the MSP manually and I set the initial value of photon index 
            to be $-2.0$.

            \subsubsection{Count Maps and Count Cubes}
              Figure \ref{fig: j1939_count_map_ave} is the comparison of count maps between 
              observation data. Like the previous conditions, the count map of observation data is 
              more messy than the fitted model. 

              \begin{figure}[!ht]
                \begin{center}
                \begin{minipage}{0.45\textwidth}
                  \begin{center} 
                    \includegraphics[scale=0.33]{/Users/grewwc/Desktop/Thesis/j1939_cmap_ave.png}
                  \end{center}
                \end{minipage}
                \begin{minipage}{0.45\textwidth}
                  \begin{center}
                    \includegraphics[scale=0.33]{/Users/grewwc/Desktop/Thesis/j1939_model_map_ave.png}
                  \end{center}
                \end{minipage}
              \end{center}
              \caption{The count maps of PSR B1937+21 created from observation 
                  data (left) and from the spectral model (right). The dimensions
                  of both figures are $141 pixels \times 141 pixels$ and each pixel's size is
                  $0.2^{\circ}\times0.2^{\circ}$.}
                \label{fig: j1939_count_map_ave}
              \end{figure}

              Figure \ref{fig: j1939_count_cube_ave} is the count maps in different energy bands.
              When energy is above about $1.6\mbox{GeV}$, there is very few photons. In many cases,
              people analyze spectra from $100\mbox{MeV}$ to $300\mbox{GeV}$ with 
              \textit{Fermi Lat}, however, the number of photon counts above $100\mbox{GeV}$ is 
              nearly negligible compared with the total counts. The total number of photon counts 
              from $100\mbox{MeV}$ to $300\mbox{GeV}$ is about $1.07\times10^7$ compared with 
              $1175$ from $100\mbox{GeV}$ to $300\mbox{GeV}$. As a result, it is reasonable to 
              use data only from $100\mbox{MeV}$ to $100\mbox{GeV}$ and there should not be any 
              noticeable difference whether I use $100\mbox{GeV}$ to $100\mbox{GeV}$ or 
              $100\mbox{GeV}$ to $300\mbox{GeV}$. The case is also true for 
              the other two pulsars --- PSRs J0218+4232 and B1821-24. 

            \begin{figure}[!htp]
              \begin{minipage}{0.32\textwidth}
                \begin{center} 
                  \includegraphics[scale=0.24]{/Users/grewwc/Desktop/Thesis/j1939_ccube_1_ave.png}
                \end{center}
              \end{minipage}
              \begin{minipage}{0.32\textwidth}
                \begin{center}
                  \includegraphics[scale=0.24]{/Users/grewwc/Desktop/Thesis/j1939_ccube_13_ave.png}
                \end{center}
              \end{minipage}
              \begin{minipage}{0.32\textwidth}
                \begin{center}
                \includegraphics[scale=0.24]{/Users/grewwc/Desktop/Thesis/j1939_ccube_19_ave.png}
                \end{center}
              \end{minipage}
              \caption{Three count maps of PSR B1937+21's count cube. The energy ranges of the 
                figures are from $100$ to $131$ MeV, from $2460$ to $3212$ MeV, from $12198$ to
                $15929$ MeV respectively.}
              \label{fig: j1939_count_cube_ave}
            \end{figure}

          \subsubsection{Binned Likelihood Analysis}
            Figure \ref{fig: j1939_count_map_diff_ave} roughly describes how well the global fit 
            is. Both left and right panel are in linear scale in order to make it easier to 
            compare. As the residual map shows, there is no obvious difference between the fitted 
            model and the observation data. The max value of the photon counts in the residual map 
            is only about $270$. 
            \begin{figure}[!ht]
              \begin{center}
              \begin{minipage}{0.45\textwidth}
                \begin{center} 
                  \includegraphics[scale=0.31]{/Users/grewwc/Desktop/Thesis/j1939_cmap_linear_ave.png}
                \end{center}
              \end{minipage}
              \begin{minipage}{0.45\textwidth}
                \begin{center}
                  \includegraphics[scale=0.31]{/Users/grewwc/Desktop/Thesis/j1939_diff_map_linear_ave.png}
                \end{center}
              \end{minipage}
              \end{center}

              \caption{The count map created from observation data (left) and residual map (right)
                of PSR B1937+21. Both left and right panels are in linear scale. 
                The dimension is $141 pixels \times 141 pixels$ and each pixel's 
                size is $0.2^{\circ}\times0.2^{\circ}$.}
              \label{fig: j1939_count_map_diff_ave}
            \end{figure}

            Table \ref{table: j1939_fit_result_ave} lists the fit results of PLExpCutoff model. 
            According to the previous work \cite{0004-637X-787-2-167}, the PLExpCutoff model is not 
            preferred over power-law model. Therefore, only the results of power-lay model is 
            reported in the paper. In this result, the PLExpCutoff model gives a TS value of 122,
            which is about the same as the previous results. 

            \begin{table}[!ht]
              \centering
                \scalebox{0.8}{
                \begin{tabular}{|ccc|}
                  \hline 
                  & This Study & Previous Results \\
                  \hline \hline 
                  Photon Index ($\Gamma$) & $2.61\pm0.22$ & $2.1\pm0.2$ \\
                  Cutoff Energy ($E_c$, GeV) & $4.90\pm2.29$ & $8\pm4$ \\
                  % \hline 
                  % Energy Flux ($10^{-11}$ erg $cm^{-2} s^{-1}$) & $4.61$ & $2.50$ & $~$ & $~$ \\ 
                  \hline
                \end{tabular}}  
                \mycaption{Fit parameters of the spectral model of PSR B1937+21. 
                  The names of parameters are consistent with Equation
                  \ref{eq: fermi_model}. The old results are from the paper \cite{0004-637X-787-2-167}.
                  Note that since the paper prefers power-law model than PLExpCutoff model, it does not 
                  report the photon flux of PLExpCutoff model.}
                \label{table: j1939_fit_result_ave}        
            \end{table} 

            Figure \ref{fig: j1939_count_spectra_ave} shows the residual counts of the global fit.
            The fit is good from $100$MeV to about $20$GeV. However, it goes significantly worse when 
            energy is too high and the residual counts are much larger than the PSRs J0218+4232 and 
            B1821-24.

            Although the TS value for PSR B1937+21 is only 122, existence in gamma-ray of the 
            target source is more obvious in the TS maps as the Figure 
            \ref{fig: j1939_tsmap_comparison_15_ave} shows. Figure 
            \ref{fig: j1939_pl_and_cutoff_ave} contains the spectra shape of the PLExpCutoff model. 

            \begin{figure}[!ht]
              \centering
              \begin{minipage}{0.40\textwidth}
                \begin{center} 
                  \includegraphics[scale=0.31]{/Users/grewwc/Desktop/Thesis/j1939_tsmap_nosource_15_ave.png}
                \end{center}
              \end{minipage}
              \begin{minipage}{0.40\textwidth}
                \begin{center}
                  \includegraphics[scale=0.31]{/Users/grewwc/Desktop/Thesis/j1939_tsmap_withsource_15_ave.png}
                \end{center}
              \end{minipage}
              \caption{TS maps of PSR B1937+21. The dimension is $3^{\circ} \times 3^{\circ}$
                ($15$ pixels $\times 15$ pixels and each pixel is 
                $0.2^{\circ} \times 0.2^{\circ}$). The left panel is generated by the model 
                by removing PSR B1937+21 while the right panel contains the target source.}
              \label{fig: j1939_tsmap_comparison_15_ave}
            \end{figure}
  

            \begin{figure}
              \centering 
              \includegraphics[scale=0.37]{/Users/grewwc/Desktop/Thesis/what_j1939_count_spectra.png}
              \caption{Top: count spectra for all sources included in the xml model (including 
                sources with both free parameters and fixed parameters). Bottom: counts residual plot
                of the model. The thick red line is the target source PSR B1937+21, the blue thick 
                lines are sources with free parameters and the thin lines are the sources with only 
                fixed parameters.}
              \label{fig: j1939_count_spectra_ave}
            \end{figure}
            

            % \add{PLExpCutoff more, below is power-law model }

            Since the fit results of PLExpCutoff model is not very satisfiable, I also fit the 
            gamma-ray spectra with a power-law model. Because I have showed lots of count maps 
            and they actually do not give us very much information, I do not show more count maps 
            and count cubes for the analysis of power-law model. The photon index is 
            $2.94\pm0.13$ with a TS value of $147$ as the Table 
            \ref{table: j1939_power_law_compare_ave} shows. Although the energy flux is consistent 
            with the previous study \cite{0004-637X-787-2-167}, the photon index is not.
            \begin{table}[!htp]
              \centering
                \scalebox{0.8}{%
                \begin{tabular}{|cccc|}
                  \hline
                  & This Thesis & Previous 1 & Previous 2 \\
                  \hline \hline 
                  Photon Index ($\Gamma$) & $2.94\pm0.13$ & $2.38\pm0.07$ & $3.02\pm0.18$ \\
                  Energy Flux ($10^{-11}$ erg$cm^{-2} s^{-1}$) & $1.6\pm0.2$ & $1.6\pm0.2$ &  \\
                  \hline 
                \end{tabular}}  
                \mycaption{Photon index comparison of power-law model between different 
                  studies. The data of column \textit{Previous 1} is from the paper 
                  \cite{0004-637X-787-2-167} and column \textit{Previous 2} is from the paper 
                  \cite{J1939_old}.}
                \label{table: j1939_power_law_compare_ave}        
            \end{table}  

            Figure \ref{fig: j1939_power_law_count_spectra} is the count residuals of the fit and 
            it is no better than Figure \ref{fig: j1939_count_spectra_ave}. But I also prefer a 
            power-law model since the likelihood of the power-law model is much larger than the 
            PLExpCutoff model. The Fermi tool \textit{gtlike} gives a likelihood value of the 
            fitting model. I can compare the absolute value of the likelihood to decide which model 
            is preferred. The absolute value for power-law model is $18409504.4$ compared with 
            $4427799.496$ for a PLExpCutoff model. And the spectrum of the power-law model is 
            shown in Figure \ref{fig: j1939_pl_and_cutoff_ave}. Both power-law and PLExpCutoff 
            models are consistent with the flux points, which are generated independently, and 
            the flux points do not show a clear cutoff. 

            \begin{figure}
              \centering 
              \includegraphics[scale=0.4]{/Users/grewwc/Desktop/Thesis/j1939_power_law_count_spectra.png}
              \caption{Top panel is counts spectra for all sources within the ROI. Bottom panel  
                shows the counts residual. }
              \label{fig: j1939_power_law_count_spectra}
            \end{figure}

            Figure \ref{fig: j1939_power_law_tsmap_withsource_20} shows the TS maps with the 
            target source and without the target source respectively. 

            \begin{figure}[!htp]
              \begin{center}
              \begin{minipage}{0.45\textwidth}
                \begin{center} 
                  \includegraphics[scale=0.38]{/Users/grewwc/Desktop/Thesis/j1939_power_law_tsmap_no_source_20.png}
                \end{center}
              \end{minipage}
              \begin{minipage}{0.45\textwidth}
                \begin{center}
                  \includegraphics[scale=0.38]{/Users/grewwc/Desktop/Thesis/j1939_power_law_tsmap_with_source_20.png}
                \end{center}
              \end{minipage}
            \end{center}
 
            \caption{TS maps of PSR B1937+21 (power-law model). The dimension is $4^{\circ} \times 4^{\circ}$
              ($20$ pixels $\times 20$ pixels and each pixel is 
              $0.2^{\circ} \times 0.2^{\circ}$). The left panel is generated by the model 
              after removing PSR B1937+21 while the right panel contains the source.}
            \label{fig: j1939_power_law_tsmap_withsource_20}
          \end{figure}

          \begin{figure}[!htp]
            \centering
            \includegraphics[scale=0.3]{/Users/grewwc/Desktop/Thesis/j1939_pl_and_cutoff_ave.png}
            \caption{The log-log plot of flux to energy of PSR B1937+21’s gamma-ray spectrum.
              The red line represents the PLExpCutoff model and the green line is the power-law
              model. }
            \label{fig: j1939_pl_and_cutoff_ave}
          \end{figure}

%%%%%%%%%%%%%%%%%%%%%%%%%%%%%%%%%%%%%%%%%%%%%%%%%%%


          \subsection{Phase Resolved Analysis}
            Since the PLExpCutoff model is not a good model for phase-averaged spectrum of 
            PSR B1937+21, I did a quick phase-resolved analysis for the MPS. I choose the 
            phase of $0.0 \sim 0.2$ and $0.5 \sim 0.7$ from the full good time interval (GTI) as 
            Figure \ref{fig: j1939_phase} implies. 
            \begin{figure}[!htp]
              \centering 
              \includegraphics[scale=0.55]{/Users/grewwc/Desktop/Thesis/b1937_parfiles.png}
              \caption{Top left panel shows the pulse phase of PSR B1937+21 in gamma-ray band. 
                The figure is produced by tempo2.}
              \label{fig: j1939_phase}
            \end{figure}

            The light curve generated by \textit{Fermi Lat} is consistent with the paper 
            \cite{J1939_old} as Figure \ref{fig: j1939_light_curve_compare} shows. 
            The new light curve by \textit{LAT} is better than the previous \textit{LAT} light curve
            as shown in the figure. 

            \begin{figure}
              \centering 
              \includegraphics[scale=0.4]{/Users/grewwc/Desktop/Thesis/j1939_lightcurve_all.png}
              \caption{Light curves of PSR B1937+21 by different telescopes. The data and figure
                is from the paper \cite{J1939_old}}
              \label{fig: j1939_light_curve_compare}
            \end{figure}


            The radius of the ROI is also set to $20^{\circ}$ degrees, and all parameters of 
            sources $8^{\circ}$ degrees outside from the center are fixed with default values, 
            which is the same as the above analysis. 
            Figure \ref{fig: j1939_count_map} is the comparison of count maps between 
            observation data and the model.

          \subsubsection{Count Maps and Count Cubes}
            \begin{figure}[!ht]
              \begin{center}
              \begin{minipage}{0.45\textwidth}
                \begin{center} 
                  \includegraphics[scale=0.38]{/Users/grewwc/Desktop/Thesis/j1939_cmap.png}
                \end{center}
              \end{minipage}
              \begin{minipage}{0.45\textwidth}
                \begin{center}
                  \includegraphics[scale=0.38]{/Users/grewwc/Desktop/Thesis/j1939_cmap_model.png}
                \end{center}
              \end{minipage}
            \end{center}
 
          \caption{The count maps of PSR B1937+21 created from observation 
              data (\textsf{left}) and from the spectral model (\textsf{right}). The dimensions
              of both figures are $141 pixels \times 141 pixels$ and each pixel's size is
              $0.2^{\circ}\times0.2^{\circ}$.}
            \label{fig: j1939_count_map}
          \end{figure}

          There are four point sources with free parameters in the model which are represented 
          by the green circles in Figure \ref{fig: j1939_count_map} and I add the PLExpCutoff model 
          for PSR B1937+21 manually. 

          Like the phase-averaged case, the count map is so messy that the source PSR B1937+21 
          is completely not identifiable in this count map.  
          Figure \ref{fig: j1939_count_cube} shows count maps in different energy bands. As 
          the figure shows, the lower energy band is very messy while there is no valuable  
          data in the high energy band. Therefore, the data in the middle energy range is 
          more reliable and when fitting the two-layer model, it has higher priority to 
          minimize the differences between the model and observation data in the middle part.

          % \add{Continue from here... \\
          %   PSR B1937+21 is a little bit different. I did a phase resolved analysis, 
          %   in order to show if it is necessary, I plan to do a new phase averaged analysis 
          %   for comparison. The best phase averaged results are not finished completely yet, 
          %   but soon.}
          \begin{figure}[!ht]
            \begin{minipage}{0.32\textwidth}
              \begin{center} 
                \includegraphics[scale=0.24]{/Users/grewwc/Desktop/Thesis/j1939_ccube_start.png}
              \end{center}
            \end{minipage}
            \begin{minipage}{0.32\textwidth}
              \begin{center}
                \includegraphics[scale=0.24]{/Users/grewwc/Desktop/Thesis/j1939_ccube_middle.png}
              \end{center}
            \end{minipage}
            \begin{minipage}{0.32\textwidth}
              \begin{center}
              \includegraphics[scale=0.24]{/Users/grewwc/Desktop/Thesis/j1939_ccube_end.png}
              \end{center}
            \end{minipage}
            \caption{Three count maps of PSR B1937+21's count cube. The energy ranges of the 
              figures are from $100$ to $126$ MeV, from $1.99$ to $2.51$ GeV, from $50.12$ to
              $63.10$ GeV respectively.}
            \label{fig: j1939_count_cube}
          \end{figure}

        \subsubsection{Binned Likelihood Analysis}
          Figure \ref{fig: j1939_count_map_diff} is a combination of the residual map the 
          original count map of PSR B1937+21 in linear scale. The fact that there are many 
          dots in the residual map implies that the global fit is not as good as the 
          other two pulsars. Furthermore, the fit is particularly bad in the cyan circle 
          region in the right bottom part Figure \ref{fig: j1939_count_map_diff}. 
          The total numbers of counts of the cyan region are $18975$ and $4212$ respectively 
          for the count map and the residual map. The center of the region is 
          $\left(290.53^{\circ}, 14.12^{\circ}\right)$ and is about $8.65^{\circ}$ away from 
          the source, which means that parameters of the sources in the region are fixed 
          during the fit. This may be the problem of the method of choosing the phase. 
          Firstly I get the pulse phase of PSR B1937+21 and select $40\%$ of the whole 
          rotation period containing the pulse phase. At the same time, I need to change the 
          scales to $40\%$ of their original values for the sources with no free 
          parameters. This is reasonable, however, for some bright sources, a little 
          difference in percent can cause big residuals of photon counts. 
          \question{Not Sure}

          \begin{figure}[!ht]
            \begin{center}
            \begin{minipage}{0.45\textwidth}
              \begin{center} 
                \includegraphics[scale=0.31]{/Users/grewwc/Desktop/Thesis/j1939_cmap_linear.png}
              \end{center}
            \end{minipage}
            \begin{minipage}{0.45\textwidth}
              \begin{center}
                \includegraphics[scale=0.31]{/Users/grewwc/Desktop/Thesis/j1939_cmap_diff_linear.png}
              \end{center}
            \end{minipage}
            \end{center}

            \caption{The count maps of PSR B1937+21 created from observation 
              data (left) and from the spectral model (right). The dimensions
              of both figures are $141 pixels \times 141 pixels$ and each pixel's size is
              $0.2^{\circ}\times0.2^{\circ}$.}
            \label{fig: j1939_count_map_diff}
          \end{figure}

          The Figure \ref{fig: j1939_count_spectra} is the combination of count spectra for 
          all sources within the $20^{\circ}$ and the count residual plot. The count residual 
          plot shows that except low gamma-ray energy part (from $100\hbox{MeV}$ to 
          $200\hbox{MeV}$) and very high energy band (above 10GeV) the total number of photon 
          counts of the model is very close to the observation data. This seems good, but can 
          also imply some problems. As previously discussed, the modeled number of photon counts 
          in the cyan circle is $7924$ smaller than the observation data. And the count residual 
          plot shows that the count number of the whole map is nearly the same, which means that 
          in the model there must be some sources forced to be larger than the real value to make 
          up for the deficiency. This implies that the fit results of some sources with free 
          parameters are not as good as the Figure \ref{fig: j1939_count_spectra} shows. 
          
          \begin{figure}[!htp]
            \centering
            \includegraphics[scale=0.42]{/Users/grewwc/Desktop/Thesis/j1939_count_spectra.png}
            \caption{The count spectra (top) of all sources including those with only fixed 
              parameters and count residuals (bottom) of the fit for PSR B1937+21.}
            \label{fig: j1939_count_spectra}
          \end{figure}

          Table \ref{table: j1939_fit_result} lists the results of the fit parameters of 
          PSR B1937+21. We see from the table that the new fit results are 
          consistent with the old ones and the precision improves a lot. 
          Figure \ref{fig: j1939_pl_and_cutoff} is a plot of the PLExpCutoff model.
          The global fit is consistent with flux points fitted by each energy bin separately 
          as the flux points and the red line with shade shows. Surprisingly, the power-law model
          is very close to the phase-resolved PLExpCutoff model. 
          The TS value of our target source is $383$, which gives 
          a significance level $\sigma \approx \sqrt{TS} = 19.6$. This implies the presence 
          of the target source. I also generate TS maps to test the presence of the source 
          as Figure \ref{fig: j1939_tsmap_comparison_15} shows. 
          
          \begin{figure}[!htp]
            \centering 
            \includegraphics[scale=0.4]{/Users/grewwc/Desktop/Thesis/j1939_pl_and_cutoff.png}
            \caption{PLExpCutoff model spectrum of PSR B1937+21 (phase resolved). }
            \label{fig: j1939_pl_and_cutoff}
          \end{figure}
          
          \begin{figure}[!htp]
            \centering
            \begin{minipage}{0.40\textwidth}
              \begin{center} 
                \includegraphics[scale=0.31]{/Users/grewwc/Desktop/Thesis/j1939_tsmap_nosource_15.png}
              \end{center}
            \end{minipage}
            \begin{minipage}{0.40\textwidth}
              \begin{center}
                \includegraphics[scale=0.31]{/Users/grewwc/Desktop/Thesis/j1939_tsmap_withsource_15.png}
              \end{center}
            \end{minipage}
            \caption{TS maps of PSR B1937+21. The dimension is $3^{\circ} \times 3^{\circ}$
              ($15$ pixels $\times 15$ pixels and each pixel is 
              $0.2^{\circ} \times 0.2^{\circ}$). The left panel is generated by the model 
              after removing PSR B1937+21 while the right panel contains the source.}
            \label{fig: j1939_tsmap_comparison_15}
          \end{figure}

          \begin{table}[!htp]
            \centering
              \scalebox{0.8}{
              \begin{tabular}{|ccc|}
                \hline 
                & This Study & Previous Results \\
                \hline \hline
                Photon Index ($\Gamma$) & $2.37\pm0.06$ & $1.43\pm0.87$ \\
                Cutoff Energy($E_c$, GeV) & $4.5\pm1.2$ & $1.15\pm0.74$ \\
                % \hline 
                % Energy Flux (erg $cm^{-2} s^{-1}$) ($10^{-12}$) & $19$ & $0.3$ & $16$ & $2$ \\ 
                \hline
              \end{tabular}}  
              \mycaption{Fit parameters of the spectral model of PSR B1937+21 (phase resolved). 
                The previous results are from the paper \cite{0004-637X-787-2-167}.
                Since the studied energy range is different, I do not list the energy flux for 
                comparison.}
              \label{table: j1939_fit_result}        
          \end{table} 

    \chapter{Theory and Simulation}
      \section{A Brief Introduction to Outer Gap Model}
        It is oversimplified to regard a pulsar as a magnetized sphere rotating in vacuum. 
        Actually, there are plenty of charged particles in a pulsar's magnetosphere 
        which co-rotate with the pulsar. The creation of charged particles can 
        be described following steps \cite{Sturrock:1971zc}.

        % The co-rotating charged primary particles emit gamma-ray by \\curvature 
        % radiation because they are accelerated by environmental magnetic fields.  
        In intense magnetic field, the high energy photons decay into electrons and 
        positrons which are called secondary particles by the process: 
        $\gamma + (B) \rightarrow e^++e^-+(B)$ and these charged particles can emit 
        synchrotron radiations. The secondary particles in charge-deficient regions can also 
        be accelerated to very high speed by strong magnetic field just like primary particles 
        and some of them then emit gamma-rays which can further decay into electrons and 
        positrons. As a result, these charged particles can create more secondary particles.
        This chain of process is quite efficient to produce charged particles and 
        make a pulsar's magnetosphere filled with plasma as a consequence. Therefore, a 
        characteristic charge density $\rho_{GJ}=-\frac{\vec{\Omega}\cdot \vec{B}}{2\pi c}$ 
        called Goldreich-Julian charge density is produced \cite{1969ApJ}. By the definition 
        of $\rho_{GJ}$, there is a surface called null charge surface where $\rho_{GJ}$ is 
        very close or equal to $0$. 

        Since the charged particles cannot move along the magnetic field lines near 
        the light cylinder, there are closed field lines and open field lines. Charged 
        particles can move out of the magnetosphere along those open field lines. Thus, the 
        charge density in the regions near the light cylinder can be much smaller than 
        \gj{} and electrons and positrons are accelerated to very high speeds by electric 
        fields parallel to the magnetic field lines ($E_{\parallel}$). These regions are called 
        outer gap which are between the null charge surface and the light cylinder.
        \cite{1986ApJ...300..500C} 
        
        Since electrons and positrons are accelerated to opposite directions, there are many 
        charged particles move toward the pulsar's surface. As discussed above, there is no 
        significant electric field to accelerate the incoming particles, they emit softer 
        photons than the photons emitted by the particles moving outward. Furthermore, since the 
        softer photons are close the stellar surface, they can generate electrons and 
        positrons with the help of high magnetic fields. These charged particles maintain the current 
        moving along the open field lines. Meanwhile, the particles moving from null charge 
        surface to the light cylinder are largely accelerated by $E_{\parallel}$, and hence 
        emit gamma-rays by curvature radiations. Part of the gamma-rays covert to
        electron-positron pairs by colliding with soft photons, and the pairs compensate 
        the deficit of charge densities, hence stop the growth of the outer gap. This is the 
        very basic introduction of the outer gap model, which is helpful to understand the 
        two-layer model. 
        
        % In short, the whole process can be crudely approximated by the following procedures. 
        % At first, charged particles move out of the magnetosphere along the open field lines
        % and forms strong electric fields parallel to the magnetic field lines 
        % ($E_{\parallel}$). Then particles with opposite sign are accelerated to very high 
        % speeds by the $E_{\parallel}$ and move to opposite directions. The incoming charges 
        % can further generate more particles with the help of the high magnetic field near the 
        % stellar surface and maintain the current moving along the open field lines. On the 
        % other hand, the charges moving outward emit curvature photons, 



      \section{Two-layer Model}
        After reviewing gamma-ray emission mechanism, we can proceed to
        the two-layer model on which this thesis is mainly based \cite{0004-637X-720-1-178}. 
        Two-layer model is a variation of outer-gap model since they both claim that the 
        gamma-ray emission zone is close to the light cylinder. However, in the two-layer 
        model, the outer layer consists of two regions --- a primary acceleration region and 
        a screening region. 

        In the primary region, charged particles moved out of pulsars along the open field 
        lines, so the charge density is usually very low. 
        However, by pair-production processes, a lot of $e^{-}$ and $e^{+}$ are produced. 
        But in the primary region where lots of pairs are created, the charge density 
        doesn't change very much because the pairs have not been separated yet. With the help 
        of strong electric field, the particles of opposite signs move to opposite directions. 
        As a result, the two-layer model states that above the primary region, a screening 
        region will be created and the charge density is very large because of the accumulation 
        of the charged particles. This is basically the reason why there are two regions in 
        pulsars' outer magnetosphere.

        The next issue is that how to describe the distribution of charge density in these two 
        regions. For simplicity, the model just uses a step function to represent the charge 
        density distribution and a step function can well describe the large charge density 
        difference between the two regions. It uses a magnetic dipole model to approximate the 
        magnetic distribution in the magnetosphere. Since by magnetic dipole model, the magnetic 
        field at one position is only dependent on the position's distance from the source 
        and altitude, the model also ignores the azimuthal distribution of charge density 
        and uses two parameters which are distance $r$ and altitude $\theta$ to calculate the 
        magnetic field at a particular position.

        The two-layer model uses three parameters to express the structure of a pulsar's outer 
        magnetosphere --- charge density of the primary region, the total height of the primary 
        region and the screening region and the last one is the ratio of the thickness of the 
        primary region and the screening region. Figure \ref{fig: charge_density} shows the 
        basic structure of two-layer model. 

        \singleFig{charge_density}{0.6}{(a): Geometry of the two-layer model. $h_{1}$ and 
          $h_{2}$ is the height of the primary region and the screening region respectively. 
          (b): the charge densities of primary region and screening region. In the primary 
          region, the charge density is much smaller than Goldreich-Julian charge density
          while is larger in the screening region. The figure is from 
          \cite{0004-637X-720-1-178}.}

        As Figure \ref{fig: charge_density} shows, 
        let the charge density of the primary region be $\rho_1 = (1-g_{1}) \rho_{GJ}$ and 
        the total gap size be
        $h_{2}$, where $\rho_{GJ}$ is the Goldreich-Julian charge density. For convenience, 
        also denote the gap size of the primary region as $h_{1}$. 
        \myComment{Then we can calculate electric potential and electric field by solving the 
        Poisson equation }
        Denote the electrical potential as $\phi_{0}$ which satisfies 
        \begin{equation}
          \label{eq: Poisson_corotating}
          \nabla^{2}\phi_{0} = -4\pi\rho_{GJ}
        \end{equation}
        and the total electrical potential is $\phi = \phi_{0} + \phi^{\prime}$, 
        where $\phi^{\prime}$ is a representation of the deviation from the co-rotating 
        electrical potential. Let the total charge density be $\rho$ and subtract by
        Equation \ref{eq: Poisson_corotating} we have,
        \begin{equation}
          \label{eq: Poisson_final}
          \nabla^{2}\phi^{\prime} = -4\pi\left(\rho - \rho_{GJ} \right)
        \end{equation}

      Because the model has ignored the distribution in the azimuthal direction, it uses two 
      parameters $x, z$ to represent a position, where $x$ is the direction along the magnetic 
      field line and $z$ is perpendicular to the magnetic field line. In order to solve 
      Equation \ref{eq: Poisson_final}, the model also makes two approximations. The first is that 
      the derivative of electrical potential $\phi$ is ignored. The second is that the \gj{} 
      is uniformly distributed along the magnetic line direction ($x$ direction). These two 
      approximations are based on a reasonable assumption that the change rate 
      for both electrical potential ($\phi^{\prime}$) and \gj{}($\rho_{GJ}$) along the $x$ 
      direction is much smaller than the $z$ direction. 
      As a consequence, Equation \ref{eq: Poisson_final} can be written as: 
      \begin{equation}
        \label{eq: Poisson_final_final}
        \frac{\partial^2}{\partial z^2} \phi^{\prime} = -4\pi\left(\rho - \rho_{GJ} \right)
      \end{equation}

      In order to solve Equation \ref{eq: Poisson_final_final}, proper boundary conditions are also 
      needed. First of all, we have to decide the boundary positions, which is determined by 
      four parameters and they can be written as $x_{lo}, x_{hi}, z_{lo}, z_{hi}$. It is 
      reasonable to set $x_{lo}$ and $x_{hi}$ be the stelar surface and the light cylinder 
      respectively and $z_{lo}$ (lower boundary) be the last open field line. And let the 
      electrical potential be $0$ along the last open field line (this is because the 
      variation of electric field strength along the $x$ direction is ignored) as Equation 
      \ref{eq: lower_boundary} shows.  

      \begin{equation}
        \label{eq: lower_boundary}
        \phi \left(x, z_{lo}\right) = 0
      \end{equation}
      To determine the position of $z_{hi}$ is a little bit tricky. In order to make the electrical 
      potential be continuous at $z = z_{hi} = h_2$, the model sets the 
      $\phi^{\prime}\left(z=h_{2}\right) = 0$ since the non-co-rotating electrical potential 
      outside the upper bound is $0$ and the co-rotating potential is continuous near the 
      boundary. Additionally, because $\phi^{\prime}\left(z=h_{2}-\right) = 0$
      and $\phi^{\prime}\left(z=h_{2}+\right) = 0$, it is known that the first derivative 
      $\partial{\phi^{\prime}}/\partial{z}\vert_{z=h_{2}}$ is $0$, which means 
      $E_{\perp}\vert_{z=h_{2}} = 0$. In order to solve Equation \ref{eq: Poisson_final_final}, 
      denote charge densities of the two regions for convenience as the function 
      \ref{eq: twolayer_charge_density} shows.
      \begin{equation}
        \label{eq: twolayer_charge_density}
          \rho\left(z\right) = 
          \begin{cases}
             & \rho_{1} , \text{    if} \left(0 \leq z < h_{1}\right)\\
             & \rho_{2} , \text{    if} \left(h_{1} \leq z \leq h_{2}\right) 
          \end{cases}       
      \end{equation}
      With definition in Equation \ref{eq: twolayer_charge_density} and the three boundary 
      conditions, the solution of Equation \ref{eq: Poisson_final_final} is, 
      \begin{equation}
        \label{eq: twolayer_potential}
          \phi^{\prime}\left(z, x\right) = -2\pi
          \left\{\begin{alignedat}{2}
             & \left(\rho_{1} - \rho_{GJ}\left(x\right)\right)z^2 + C_{1} z ,  &&\left(0 \leq z < h_{1}\right)\\
             & \left(\rho_{2}-\rho_{GJ}\left(x\right)\right)\left(z^2 - h_2^2\right) + D_{1} \left(z-h_2\right),  &&\qquad \left(h_{1} \leq z \leq h_{2}\right) 
          \end{alignedat}\right.
      \end{equation}
      where 
      \begin{equation*}
        C_{1} = \frac{\left(\rho_{1}-\rho_{GJ}\left(x\right)\right)h_1\left(h_1-2h_2\right)-\left(\rho_2-\rho_{GJ}\left(x\right)\right)\left(h_1-h_2\right)^2}{h_2} 
      \end{equation*}
      and 
      \begin{equation*}
        D_{2} = \frac{\left(\rho_1-\rho_2\right)h_1^2-\left[\rho_2-\rho_{GJ}\left(x\right)\right]h_2^2}{h_2}
      \end{equation*}
      From Equation \ref{eq: twolayer_potential} and apply 
      $\rho_{GJ}\left(x\right)=-\left(\Omega B x\right)/\left(2\pi cs\right)$ 
      we can directly derive the strength of electrical fields parallel to magnetic field lines as a 
      function of $z$ as Equation \ref{eq: twolayer_field} shows.
      \begin{equation}
        \label{eq: twolayer_field}
          E^{\prime}_{\parallel}\left(z\right) = \frac{\Omega B}{cs}
          \left\{\begin{alignedat}{2}
             & -g_1 z^2 + C_1^{\prime}z ,  &&\left(0 \leq z < h_{1}\right) \\
             & g_2\left(z^2 - h_2^2\right) + D_1^{\prime}\left(z-h_2\right)  &&\qquad \left(h_{1} \leq z \leq h_{2}\right) 
          \end{alignedat}\right.
      \end{equation}
      where 
      \begin{equation*}
        C_{1}^{\prime} = \frac{g_1 h_1 \left(h_1 - 2h_2\right)+ g_2\left(h_1-h_2\right)^2}{h_2}  
      \end{equation*}
      and 
      \begin{equation*}
        D_{2}^{\prime} = -\frac{\left(g_1 + g_2\right)h_2^2 + g_2 h_2^2}{h_2}
      \end{equation*}

      Since charged particles are accelerated in the primary region to relativistic speeds 
      and then emit energies by curvature radiations, we have
      \begin{equation}
        \label{eq: all_is_curvature_radiation}
        e E_{\parallel}^{\prime} c = l_{cur}
      \end{equation}
      where $E_\parallel^{\prime}$ is the electric field strength along the magnetic field 
      line described in Equation \ref{eq: twolayer_field}.
      Lorentz factors of the charged particles is estimated according to Equation 
      \ref{eq: curvature_radiation_power}.
      \begin{equation}
        \label{eq: curvature_radiation_power}
        l_{cur} = \frac{2 e^2 c \gamma^{4}_{e}}{3s^2}
      \end{equation}      
      where $s$ is the radius of curvature. 
      Combining Equation \ref{eq: all_is_curvature_radiation} and Equation
      \ref{eq: curvature_radiation_power} we get expression for Lorentz factor: 
      \begin{equation}
        \label{eq: gamma_can_be_zero}
        \gamma_{e} = \left(\frac{3s^2}{2e} E_{\parallel}^{\prime}\right)^{1/4}
      \end{equation}
      With Lorentz factor of a charge particle known, we can derive curvature radiation 
      spectrum for single charged particle and then can get the total spectrum by integrating 
      over all charged particles. This is the simplified idea of the Two-layer model. 
      

      \section{Numerical Calculation of Spectra Based on Two-layer Model}
        After understanding the theory part of the two-layer model, 
        I can then do numerical calculations of the spectra for the three MSPs
        based on the theory. Since the theory is the same for the three MSPs, the 
        calculation algorithms are the same. 

        There are three independent parameters to fit altogether in the calculation. 
        The first parameter is fractional gap size $f=h_{2}/R_{lc}$, where $h_{2}$ is the 
        total gap size including both the primary acceleration region and the screening region 
        and $R_{lc}$ is the radius of the light cylinder. The second parameter is $g_{1}$ so that
        the charge density in the primary accelerating region is $\left(1-g_{1}\right) \rho_{GJ}$, 
        where $\rho_{GJ}$ is the Goldreich-Julian charge density. The third parameter is the ratio 
        between the sizes of the two gaps ($h_{1}/h_{2}$). Note that I only need to set the charge 
        density in the primary acceleration region as an independent parameter, since the charge 
        density in the two gaps are related to each other. Figures 
        \ref{fig: j0218_twolayer_cur.png}, \ref{fig: b1821_twolayer_cur.png} and 
        \ref{fig: j1939_twolayer_cur.png}
        are the spectra of the three MSPs generated from the two-layer model and the results of the 
        fit parameters are listed in Table \ref{table: twolayer_fit_parameter}. Other than the 
        low energy and high energy gamma-ray bands, the model is consistent with the observation 
        data in terms of gamma-ray part. 

        Generally speaking, the modeled spectra for the three MSPs are acceptable. Just as 
        we have discussed in the data analysis part that global fits are not very consistent 
        with separate fits in low energy gamma-ray band (about $100\mbox{MeV} - 500\mbox{MeV}$) 
        and high energy band (above $10 \mbox{GeV}$), the modeled spectra also have the same 
        problem. This can have two explanations. Firstly, the Fermi telescope is not sensitive 
        below about $100 \mbox{MeV}$. As a result, the observation data may not be very precise
        at this energy band. Secondly, the real emission mechanism in the energy band is different 
        from the model describes. Thus, the obvious inconsistencies between the calculations 
        and observations can be observed.
 
        \singleFig{j0218_twolayer_cur.png}{0.37}{The gamma-ray spectrum of PSR J0218+4232 by 
          the two-layer model.}
        \singleFig{b1821_twolayer_cur.png}{0.37}{The gamma-ray spectrum of PSR B1821-24 by 
          the two-layer model.}
        \singleFig{j1939_twolayer_cur.png}{0.38}{The gamma-ray spectrum of PSR B1937+21 by 
          the two-layer model.}

        % \mayChange{I wanted to combine the three figures together, if I do so, each 
        %   figure is too small.}
        
        \begin{table}[!htp]
          \centering
          \scalebox{0.8}{
          \begin{tabular}{|cccc|}
            \hline
            & f & g & $h_1/h_2$ \\ 
            \hline \hline
            PSR J0218+4232 & 0.330 & 0.92 & 0.915 \\
            PSR B1821-24 & 0.247 & 0.955 & 0.920 \\
            PSR B1937+21 & 0.320 & 0.975 & 0.925 \\
            \hline 
          \end{tabular}}
          \mycaption{The results of fitting parameters for the three MSPs. The physical 
            meaning of each parameter is consistent with the two-layer model described above.}
            % \change{Table is ugly, but not sure how to make it more beautiful...}}
          \label{table: twolayer_fit_parameter}
        \end{table}
        \vspace{0.5cm}
          
        After obtaining the spectra fit results in gamma-ray band, I then generate broad band 
        spectra as Figures \ref{fig: j0218_twolayer_all.png}, \ref{fig: b1821_twolayer_all.png}
        and \ref{fig: j1939_twolayer_all_ave.png} shows. 
        The hard X-ray data are from the paper \cite{0004-637X-845-2-159}. By tweaking the 
        independent parameters of the two-layer model, I can make the modeled spectra very 
        close to the observation data in hard X-ray band. Since the lack of observation data in 
        the energy band from about $100$keV to $100$MeV, it is hard to tell if the two-layer model 
        describes the right physical scenario in this energy range. However, the prediction made 
        by the simplified two-layer model is generally precise. In addition, the model is very 
        intuitive, which is also a very important consideration for building a model. Just as 
        the famous words "With four parameters I can fit an elephant, and with five I 
        can make him wiggle his trunk" 
        said by John von Neumann, in principal, we can fit any data by adding independent 
        parameters. Therefore, in order to test if a theory model is good or not, not only we 
        need to consider how precisely the model can predict, but also the physical meaning 
        behind the model. In this sense, the two-layer model is a good start of explaining 
        emission mechanism of pulsars. 
          
        \singleFig{j0218_twolayer_all.png}{0.37}{The broad band and modeled spectrum of PSR J0218+4232.
          The grey shade is the error of the global fit. And the green shade in the left panel ef the 
          figure represents the error of hard X-ray. The 'Total' legend represents the total flux 
          combining the Synchrotron radiation, inverse Compton radiation and curvature radiation
          altogether.}
        \vspace{0.5cm} 
        
        \singleFig{b1821_twolayer_all.png}{0.37}{The broad band and modeled spectrum of PSR B1821-24.
          The meaning of grey shade and the green shade are the same as Figure \ref{fig: j0218_twolayer_all.png}}
        \vspace{0.5cm} 
          
        \singleFig{j1939_twolayer_all_ave.png}{0.39}{The broad band and modeled spectrum of PSR B1937+21.
          The meaning of grey shade and the green shade are the same as Figure
          \ref{fig: j0218_twolayer_all.png}}
        \vspace{0.5cm}        

      \section{Considerations of Doing Numerical Calculation}
        \subsection{Correctness of Computation}
          To make sure the numerical computation be right is the most important. 
          The first consideration is underflow and overflow of floating digits.
          One possible condition is calculating speed of relativistic charged particles with 
          Lorentz factor $\gamma$. By doing some simple test, I find that for 
          $\gamma < 1.5\times 10^7$, the results are precise enough. However, there are 
          significant rounding errors when $\gamma > 5\times 10^7$, which means that the 
          results might be wrong for highly energetic particles if I directly use the formula 
          $\beta = \sqrt{1 - 1/\gamma^2}$.
          Likely, in the two-layer model, nearly all particles have $\gamma < 1\times 10^7$. 
          Furthermore, there are nearly no situations where double precision floating digits
          cannot handle calculation results of the two-layer model. Thus, as long as using 
          64-bit floating digits instead of 32-bit floating digits, we are free from overflowing 
          and underflow troubles. 
            
          There are some cases when a whole function can be calculated while some parts of them 
          are not. Take Function $f\left(x\right) = x\times1/x$ for example. When $x$ is too 
          large, it can not be expressed by a computer and multiplication is not associative when 
          doing floating point operation. I encounter some situations like this.
          The formula of curvature radiation spectrum contains modified Bessel function of order 
          $5/3$. In order to speed up the program, I use a polynomial to express the Bessel 
          function, as Equation \ref{func: polynomial_appro} shows. 
          \begin{equation}
            K_{5/3} \left(x\right) \simeq a \left(\frac{1}{x} + b\right)^{-cx - 1/3} \sqrt{\frac{\pi}{2}} e^{-x - d} \sqrt{x + d} %
            \left[1 + \frac{55}{72\left(x + d\right)} - \frac{10151}{10368}\left(x+d\right)^2\right] 
            \label{func: polynomial_appro}
          \end{equation}
          where $a,b,c,d$ are just positive constants and $c = 0.96 < 1$. As a result, 
          the part $(1/x + b)^{-cx - 1/3}$ in Function \ref{func: polynomial_appro} is infinity
          when $x$ is large though the total function is approximated to $0$. Thus, I have to 
          explicitly assign the result to be $0$ instead of calculating it. Actually, this error 
          is not easy to find since in most cases the results are not infinity. 
          
        \subsection{Speed of Computation}
          I actually have not done any accurate benchmarks for the following discussions and they are 
          dependent of the average time of running the simulations.
          The most obvious solution is to use multicores to do computation. However, most library functions 
          do not support run concurrently and only run on a single core. For example, I need to do many 
          integrations and the speed of integration is critical. I write some simple functions to 
          utilize four CPU cores at the same time when doing integration. 
          This gives me a huge performance improvement.

          Furthermore, There are some facts about the basic operations. For instance, add is 
          faster than multiplication which is faster than devision. Multiplications and devisions 
          are not associative between floating points. Though the performance differences between 
          different operations for integers can usually be optimized away by modern compilers, 
          the compilers can do nothing for floating points. Thus, I have to do it by ourselves. 
          For example, I have $z^2 - h_2\left(x\right)^2$ in function \ref{eq: twolayer_field}.
          In this formula, there are multiplications and one subtraction. After re-writing it to 
          $\left(z-h_2\right)\left(z + h_2\right)$, we have one addition, one subtraction and one 
          multiplication. Since addition and subtraction is not slower than multiplication, it 
          has no performance harm by the rewriting. What need to be noticed is that the 
          multiplication may not be slower than addition
          and it is dependent on machines. However, division is definitely slower than the other three operations. 
          Therefore, in our program, expressions like $1 / 3$ are rewritten to $1*0.3333$ and so on.  
  %stupid
      
    \chapter{Conclusion}
      The spectra of PSRs J0218+4232, B1937+21 and B1821-24 are studied using more observation 
      data and newly published \textit{Pass 8} dataset. The new results of the thesis are 
      generally consistent with the old results as the Table \ref{table: final_results} shows. 
      For PSR B1937+21, I also list the results of power-law model as Table 
      \ref{table: final_results_j1939_pl} shows.
      \begin{table}[!ht]
        \centering
          \resizebox{\textwidth}{!}{%
          \begin{tabular}{|c|c|c|c|c|c|c|}
            \cline{1-7}
            & \multicolumn{2}{|c|}{J0218+2134} & \multicolumn{2}{|c|}{B937+21} & \multicolumn{2}{|c|}{B1821-24} \\
            \cline{1-7} 
            & New & Previous & New & Previous & New & Previous \\
            \cline{1-7}
            Photon Index $\Gamma$ & $1.89\pm 0.04$ & $2.0\pm0.1$ & $2.61$ & $1.43\pm0.87$ & $1.91\pm0.07$ & $1.6\pm0.3$  \\
            Cutoff Energy (GeV)& $3.77\pm0.40$ & $4.6\pm1.2$ & $4.90\pm0.30$ & $1.15\pm0.74$ & $4.50\pm0.71$ & $3.3\pm1.5$\\
            Photon Flux ($10^{-8}$ $cm^{-2} s^{-1}$) &$7.29\pm0.28$ &$7.7\pm0.7$ &$3.76\pm0.35$ & ~ &$3.85\pm0.31$ & $1.5\pm0.6$ \\
            Energy Flux ($10^{-11}erg$ $cm^{-2} s^{-1}$) & $4.45\pm0.16$ & $4.56\pm0.24$ & $1.58\pm0.15$ & ~ & $2.44\pm0.14$ & $1.3\pm0.2$ \\
            \hline 
          \end{tabular}}  
          \mycaption{Fit parameters of the spectral model of PSR J0218+4232. 
            The names of parameters are consistent with Equation
            \ref{eq: fermi_model}. The previous results are from the paper \cite{0067-0049-208-2-17}.}
          \label{table: final_results}        
      \end{table}  

      \begin{table}[!ht]
        \centering 
        \scalebox{0.8}{
          \begin{tabular}{|ccc|}
          \hline 
          & New Results & Previous Results \\ 
          \hline \hline 
          Photon Index ($\Gamma$) &  $2.94\pm0.13$ & $2.38\pm0.07$ \\
          Photon Flux ($10^{-8}$ $cm^{-2} s^{-1}$) & $5.89\pm0.68$ & $2.93$ \\
          Energy Flux ($10^{-11}erg$ $cm^{-2} s^{-1}$) & $1.92\pm0.18$ & $ 1.6\pm0.2$ \\ 
          TS value & 147 & 112 \\
          \hline 
        \end{tabular}}
        \caption{Fit results of power-law model for PSR B1937+21. Since the previous paper 
          does not show the photon flux, I cannot calculate the error bars of photon flux for 
          the previous value.}
        \label{table: final_results_j1939_pl}
      \end{table}

      With \textit{Pass 8} data and more observation data, the new results have smaller error bars
      in terms of photon index and energy flux. Further more, we have much larger TS values for 
      PSRs J0218+4232 and PSRs B1821-24 as the Table \ref{table: ts_value_compare} shows.
      For PSR B1937+21, a power-law model should be preferred because of the following reasons. 
      The $-log(likelihood)$ of PLExpCutoff model and power-law model are $-4410944$ and 
      $-18409504$ respectively, the $TS_{cut} = 2\Delta log(likelihood) = -2.86 < 9$,  thus I 
      follow the convention of the paper \cite{2013ApJS..208...17A}. 

      \begin{table}[!ht]
        \centering
          \scalebox{0.8}{
          \begin{tabular}{|cccc|}
            \hline 
            & J0218+4232 & B1821-24 & B1937+21 \\ 
            \hline 
            Previous & 1313 & 76 & 112 \\
            New Results & 6809 & 941 & 147 \\
            \hline 
          \end{tabular}}  
          \mycaption{TS values comparison between the new results and the previous results for 
            the three MSPs. }
          \label{table: ts_value_compare}        
      \end{table}  
    
      Although the two-layer model introduced in the thesis is simplified very much, its prediction
      of broad-band spectra (including hard X-ray and gamma-ray) for the three energetic MSPs are 
      generally acceptable, except that for the energy of a few hundred MeV.

      \chapter{Discussion and Future Work}
      \section{Constraints of The Two-layer Model Used In The Thesis} 
        The simplified two-layer model is consistent with observation data to some extent. 
        (The relevant data can be found in the paper \cite{0004-637X-720-1-178})
        The model uses four parameters to get a fairly good prediction of the gamma-ray 
        spectra of many pulsars. And all these four parameters have very obvious physical 
        meanings. However, the problems of the model is clear --- it is somewhat 
        oversimplified. Although there are other more sophisticated versions of the 
        two-layer model such as three-dimensional two-layer model 
        \cite{doi:10.1111/j.1365-2966.2011.18577.x} and I used the simpler one, which may cause 
        some inconsistencies between the simulations and observations. 

        Therefore, we can briefly analyze which part is oversimplified and can be improved. 
        First of all, we directly use a step function to describe the charged particle 
        distribution. Though the charge density of the screening region is much larger than 
        the primary region, using a step function is not very physical and may exaggerate 
        the change rate of charge density. At the same time, the dramatic change of charge 
        density also brings some numerical instabilities.

        Secondly, the model sets the total of screening region and primary region to be 
        rectangular shape. Though the actual shape is not clear, it should not be a 
        rectangular in theory and may be very different. In numerical simulation, changes 
        in shape of the regions will directly lead to a different integration region, 
        which may change the simulated spectra.

        Thirdly, there are some inconsistencies in the model itself according to the its 
        assumptions. According to Equation \ref{eq: curvature_radiation_power} and Equation 
        \ref{eq: gamma_can_be_zero}, since $E_{\parallel}^{\prime}$ can be $0$, we know that 
        $\gamma_{e}$ can also be $0$, which is absolutely non-physical. Although this may not 
        have large influences on the spectra, it is the problem that should be avoided.

        All in all, the model is simple and the gamma-ray spectra computed based on the model 
        is consistent with observation data. There are many much more sophisticated 
        two-layer model which are generalizations of model used in the thesis. Those models 
        may have addressed the problems described above, but the model used in the thesis 
        do have some defects. 
    
    \section{Inconsistency Between the Two-layer \\ Model and Fit Results}
      As Figures \ref{fig: j0218_twolayer_all.png}, \ref{fig: j1939_twolayer_all_ave.png} and 
      \ref{fig: b1821_twolayer_all.png} show, the two-layer model and the global fit for all 
      the three MSPs are not consistent with each other in lower gamma-ray bands (about $100$ 
      MeV). I have tried different reasonable parameter combinations to fix the problem, but 
      it is not solved. I think there could be some reasonable explanations. First of all, 
      the energy resolution of \textit{Fermi Lat} in a few hundred MeV is not as good as in 
      other gamma-ray bands. Secondly, energy band of several hundred MeV, the signal-to-noise 
      is low compared with high energy part. By directly analyzing the events files, we find that 
      about $90\%$ of photons are in $100$MeV to $1000$MeV and the background is too bright 
      compared with the target sources. Finally, the two-layer model used in the thesis are 
      oversimplified. In most cases, the spectrum produced by the theory model is not monotone and 
      curved very much. However, the global fits of the PLExpCutoff model are usually very 
      flat from $100$MeV to $500$MeV. This means that even though regardless of physical meanings,
      no matter what parameter combinations I choose, I can hardly get the similar shape in this 
      energy range. Therefore, I think the simplified version of two-layer model used in the 
      thesis (there are other more sophisticated two-layer models) cannot do a precise prediction 
      in this energy range. 
      

    \section{Analysis With \textit{LAT 8-year Point Source List}}
      The latest preliminary \textit{LAT 8-year Point Source List} (\textit{FL8Y}) is released 
      on 03 May 2018. Since the release date is a little bit late and the point source list is a 
      preliminary version, I have not finished all the spectra analysis with the data. However, 
      it is reasonable to show some results I have done with the new source list together with my
      further plan.

      Some spectra models are changed, for instance, the expression of PLSuperExpCutoff model has 
      been changed as Equation \ref{eq: new_fermi_model} shows \cite{newFermiModel}. 
      \begin{equation}
        \frac{dN}{dE} = K\left(\frac{E}{E_0}\right)^{\Gamma} e^{\left(a\left(E_0^b-E^b\right)\right)} 
        \label{eq: new_fermi_model}
      \end{equation}
      Particularly, Fermi tools combined the $E_0^b$ with $K$ as Equation 
      \ref{eq: new_fermi_model_modified} shows and the model is called PLSuperExpCutoff2. 
      \begin{equation}
        \frac{dN}{dE} = K\left(\frac{E}{E_0}\right)^{\Gamma} e^{\left(-aE^b\right)} 
        \label{eq: new_fermi_model_modified}
      \end{equation}
      In principal, there is no difference between the Equations \ref{eq: fermi_model} and 
      \ref{eq: new_fermi_model_modified}, however, the parameters needed to be fitted is different. 
      Table \ref{table: new_results_all_three} lists the fit parameters for all of the three MSPs.
      \begin{table}[!htp]
        \centering
          \scalebox{0.8}{
          \begin{tabular}{|cccc|}
            \hline
            & J0218+4232 & B1821-24 & B1937+21 \\
            \hline \hline 
            Photon Index ($\Gamma$) & $1.77\pm0.07$ & $1.14\pm0.02$ & $1.84\pm0.03$ \\
            Expfactor (a, $10^{-3}$) & $6.73\pm0.86$ & $11.76\pm0.12$ & $6.77\pm0.25$ \\
            Scale ($E_0$) & $821.48$ & $1128.68$ & $1901$ \\ 
            Index2 (b) & $0.67$ & $0.67$ & $0.67$ \\
            Photon Flux ($10^{-8}$ $cm^{-2} s^{-1}$) & $7.44\pm0.32$ & $2.44\pm0.08$ & $3.00\pm0.19$ \\
            Energy Flux ($10^{-11}$ erg $cm^{-2} s^{-1}$) & $4.45\pm0.16$ & $2.08\pm0.04$ & $1.69\pm0.06$ \\
            TS value & $7189$ &  $980$  & $149$ \\
            \hline
          \end{tabular}}  
          \mycaption{Fit results of PSRs J0218+4232, B1821-24 and B1937+21 with 
            LAT 8-year Point Source List.}
          \label{table: new_results_all_three}        
      \end{table}  

      The TS values for all the three MSPs are all a little bit larger as Table 
      \ref{table: new_ts_compare} shows. 
      \begin{table}[!htp]
        \centering
          \scalebox{0.8}{
          \begin{tabular}{|cccc|}
            \hline
            & J0218+4232 & B1821-24 & B1937+21 \\
            \hline \hline 
            FL8Y & $7189$ & $980$ & $149$ \\ 
            3FGL & $6809$ & $941$ & $122$\\ 
            \hline
          \end{tabular}}  
          \mycaption{TS values comparison between 3FGL (older) and FL8Y (newer) source list.}
          \label{table: new_ts_compare}        
      \end{table}  



      \add{continue from here}


% \begin{thebibliography}{9}


% \end{thebibliography}
\bibliographystyle{plain}
\bibliography{bibfile}
          

\end{document}





