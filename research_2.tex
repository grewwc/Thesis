\documentclass[12pt]{report}
\usepackage{geometry}	
\usepackage[utf8]{inputenc}
\usepackage{amsmath}
\usepackage{multicol}
\usepackage{titlesec}
\usepackage{graphicx}
\usepackage{wrapfig}
\usepackage{textcomp}
\usepackage{caption}
\usepackage{subcaption}
\usepackage{comment}
\usepackage{etoolbox}
\usepackage{caption}
\usepackage{anyfontsize}
\usepackage{caption}
\usepackage{color}
\usepackage{url}
\usepackage{hyperref}

\hypersetup{
    colorlinks,
    citecolor=black,
    filecolor=black,
    linkcolor=black,
    urlcolor=black
}


\captionsetup[figure]{labelfont=bf, font=footnotesize}


%this several lines is for: no number before suction. (This is a bug)
\makeatletter
\patchcmd{\ttlh@hang}{\parindent\z@}{\parindent\z@\leavevmode}{}{}
\patchcmd{\ttlh@hang}{\noindent}{}{}{}
\makeatother


\geometry{
	a4paper,
	total={180mm,257mm},
 	left=15mm,
 	top=15mm,
 	right=15mm,
}

\title{\textbf{Thesis Title}\\ \vspace{1cm}
			{\large Department of Physics, The University of Hong Kong, Pokfulam Road, Hong Kong}\\ \vspace{1cm}
			{\includegraphics[scale=0.2]{{/home/wwc129/Desktop/Thesis/hku.png}}}\\ \vspace{3cm}
}
\date{}
\author{Wang Wenchao  \\3030053350}
\setlength{\columnsep}{1cm}

\titleformat
{\chapter} % command
[display] % shape
{\bfseries\Large} % format
{\textit{Chapter \thechapter}} % label
{0.5ex} % sep
{
    \rule{\textwidth}{1pt}
    \vspace{1ex}
    \centering
} % before-code

\titleformat{\section}[hang]
{\large\bfseries}
{\thesection}
{0.5em}
{}

\titleformat{\subsection}[hang]
{\fontsize{12}{15}\bfseries\sffamily}
{\thesubsection}
{1em}
{}

\titleformat{\subsubsection}[hang]
{\fontsize{11}{15}\bfseries\sffamily}
{\thesubsection}
{0.5em}
{}

\titleformat{\subsubsubsection}[hang]
{\fontsize{9}{15}\bfseries\sffamily}
{\thesubsection}
{0.5em}
{}



\newcommand{\mycaption}[1]{\caption{\textit{\footnotesize #1}}}
%Below is the main content.

\begin{document}
\maketitle
\tableofcontents{}
   %\vspace{1cm}
    %chapter 'Abstract'.
\begin{abstract}
    \normalsize
    Recent observations find that some millisecond pulsars (known as Class II MSPs) show aligned pulse profil
    es in 
    different energy bands. Conventionally, radio and gamma-ray emission are produced in different 
    regions---in polar cap 
    and outer gap respectively. The finding of Class II MSPs implies that radio, X-ray and gamma-ray 
    emission 
    can all be emitted in outer gap. This means that Class II MSPs can have a different emission mechanism. 
    Recently, 
    scientists propose a model suggesting that hard X-ray can be emitted by inverse Compton scattering 
    between radio
    waves and energetic charged particles. The objective of the thesis is to test this model by measuring 
    hard X-ray 
    spectra of some Class II MSPs using \textit{NuSTAR}. 
\end{abstract}
			
		
		
	%chapter 'Introduction'.
\chapter{Introduction}   	   
    \section{Neutron Stars and Pulsars}
        Neutron stars are produced by a supernova explosion of massive stars which have about 4 to 8 
        solar mass. After 
        the supernova explosion, the star leaves a central region. And the central region collapses because 
        of the effect of 
        gravity until protons and electrons combine to form neutrons ($e^{-}+p\rightarrow n+\nu_{e}$)
        ---the reason why they are called 
        ``neutron stars''.  
        Because neutrons have no electromagnetic force on each other, they can be squeezed very tightly. 
        Therefore, a neutron  
        star has tremendous high density (about $5\times 10^{17} kg/m^3$) and its diameter and mass is about
        20km and 
        1.4 solar mass respectively. What
        prevents a neutron star to continue to contract is the degeneracy pressure of neutrons. \\
        \indent Pulsars are fast-spinning neutron stars. Their rotational periods can be from a few 
        milliseconds
        to several seconds. For example, the rotational period of PSR B1937+21 is about 1.56$ms$ while 
        PSR B1919+21 is approximately 1.34$s$. As we know, a star can be ripped by centrifugal force if the
        star rotates too fast. We can estimate lower limit of density of a star with the equation: 
        $\rho=\frac{3\pi}{P^2G}$, where $P$ is the rotational period of a pulsar. Just for simplicity, we
        let $P$ be 1$s$. Then we get $\rho\approx 1.4\times 10^{11}kg/m^3$. With the knowledge  that the 
        density of a white dwarf is about $1\times 10^9kg/m^3$ which is smaller than the lower density limit,
        the observed fast-spinning stars belong to the kind of stars which are much denser 
        than white dwarf. As a result, neutron stars are ideal candidates for pulsars. \\
        \indent 
        More than 2000 pulsars have been found so far. Most of them are in the disk of our Galaxy while we also can 
        find a small portion of them in high latitude, which can be seen clearly in the figure \ref{fig:spatial 
        distribution}. This may 
        because they cannot escape the gravitational potential if their kinetic energy is not large enough. Besides,
        even though they have large enough velocities to escape from their birth region, there are some 
        probabilities that they become nearly non-detectable before reaching high latitude. 
        
        \begin{figure}[h]
          \centering
          \includegraphics[scale=0.35]{/home/wwc129/Desktop/Thesis/pulsar_distribution.png}
          \caption{\textit{\footnotesize Spatial distribution of some pulsars in galactic coordinate system.}}
          \label{fig:spatial distribution}
        \end{figure}
        


        %can add these content later.
        \iffalse 
        \indent Although the primary focuses of this thesis are observational characteristics such as 
        spectra and light-curves, it is helpful to talk a little about internal structure of pulsars.
        \fi 

        \iffalse
        \indent Many stars are in binary systems so neutron stars can also be isolated or in binary systems. 
        Some of them even have planets. That a pulsar is in binary system provides us a convenient way to 
        measure its mass. 
        \indent 
        \fi
		     
    
             


    \section{Emission Mechanism of Pulsars}
            Although emission mechanism of pulsars has not been fully understood yet, some models are developed 
            trying to 
            explain observational facts. The following is one toy model that can explain some basic features of 
            pulsars.
       
        \subsection{Magnetic Dipole Model}
            Assume a pulsar has a magnetic dipole moment $\vec{m}$, the angel between rotation axis and 
            direction of 
            $\vec{m}$ is $\alpha$, its angular velocity is $\Omega$, radius is R and moment of inertia is $I$. 
            Also by 
            assuming that energy of electromagnetic radiation are all from rotational energy, its spin-down 
            rate can be written 
            as: 
            $$
                \dot{\Omega}=-\frac{B_p^2 R^6 \Omega^3 \sin{\alpha}^3}{6c^3I}
            $$
            where $B_p$ is magnetic field strength in the pole of the pulsar. Its surface magnetic field can 
            also be estimated
            by:
            $$
                B_s=\sqrt{\frac{3c^3I}{2\pi^2R^6}P\dot{P}}=3.2\times 10^{19}\sqrt{P\dot{P}}
            $$
            where $B_s$ is the strength of surface magnetic field. \\
            \indent In general, a pulsar's spin down rate can be expressed as: $\dot{\Omega}=-K\Omega^{n}$, 
            where K is a 
            constant and n is called braking index. In magnetic dipole model n is 3 \hypersetup{urlcolor= red}
            (\href{https://arxiv.org/pdf/1506.04605.pdf}{\textit{H.Tong 2015}}). Then 
            characteristic age of the pulsar can be defined as: $P/2\dot{P}$ in magnetic dipole model. 
            For example, the Crab 
            pulsar's rotation period is about $0.033s$ and period derivative is $4.22\times 10^{-13}s/s$. 
            The characteristic 
            age is about 1200 years. The pulsar is remnant of a supernova which is observed by ancient 
            astronomers in 1054 
            AD, so the record shows that characteristic age can give us order of magnetic estimate of a 
            pulsar's real age. \\
            \indent 
            Although braking index is 3 in magnetic dipole model, most of pulsars' braking index is less than 3 as 
            shown in figure \ref{fig:braking_index}. The reason is that if a pulsar's spin down is completely because
            of pulsar wind, the braking index is 1. Thus, the real braking index should be a combination of 1 and 3,
            which is usually less than 3  \hypersetup{urlcolor= red}
            (\href{http://www.ift.uni.wroc.pl/~csqcdiv/talks/26092014/ohamil_csqcdiv.pdf}
            {\textit{Oliver Hamil 2014}}).
            \begin{figure}[!h]
              \centering
              \includegraphics[scale=0.6]{/home/wwc129/Desktop/Thesis/table.png}
              \mycaption{Braking index of some pulsars.}
              \label{fig:braking_index}
            \end{figure}
    
          
        \subsection{A More Sophisticated Model}
              It is oversimplified to regard a pulsar as a magnetized sphere rotating in vacuum. Actually,
              there are plenty of 
              charged particles in a pulsar's magnetosphere which corotate with the pulsar. The creation of 
              charged particles can 
              be described by the following steps 
              (\href{http://articles.adsabs.harvard.edu/cgi-bin/nph-iarticle\_query?1971...164..529S&amp;data
              \_type=PDF\_HIGH&amp;whole\_paper=YES&amp;type=PRINTER&amp;filetype=.pdf}{\textit{P.A.Sturrock
              1971}}). \\ 
              \indent 1. The corotating charged primary particles emit gamma-ray by curvature radiation 
              because of acceleration in super strong magnetic field.  \\
              \indent 2. In super intense magnetic field,  the high energy photons decay into electrons and 
              positrons which are called secondary particles by the process: 
              $\gamma + (B) \rightarrow e^++e^-+(B)$. Syncrotron 
                                  photons can be emitted by these secondary particles. \\
              \indent 3. Secondary particles are also accelerated in strong magnetic field which is just like 
              primary particles. As a result, these charged particles can create more secondary particles. \\
              \indent This chain of process is quite efficient to produce charged particles and pulsar's 
              magnetosphere is filled with plasam as a consequence. So, it is natural to think of the 
              distribution of charges in pulsar's magnetosphere. A characteristic charge density 
              $\rho_{GJ}=-\frac{\vec{\Omega}\cdot \vec{B}}{2\pi c}$ is called Goldreich-Julian density. 
              This charges can offset part of electric field ($E_{\parallel}$) which is parallel to magnetic 
              field. There is some region in the magnetosphere called ``outer gap'' where $\rho_{GJ}$ is so 
              small that it can't screen $E_{\parallel}$ effectively. As a result, the secondary particles can 
              be accelerated at a very large velocity (Lorentz factor $\gamma\sim 10^7$) and emit gamma-ray. 
              Photons in outer gap can also create electrons and positrons by the process: 
              $\gamma+\gamma\rightarrow e^-+e^+$. At the outer gap, one-photon pair production can't happen 
              because magnetic field is too weak in this region.
            
                                
        
             

    \section{Millisecond Pulsar} 
        \subsection{P-$\dot{\mathbf{P}}$ Diagram} 
        P-$\dot{\mathrm{P}}$ diagram is an important tool for analyzing evolution of pulsars. 
        Period (P) and time derivative of period ($\dot{\mathrm{P}}$) are two of pulsars' important 
        characteristics. Analyzing the position of a pulsar in P-$\dot{\mathrm{P}}$ diagram can give some 
        valuable information such as which evolution stage the pulsar is in or the type of the pulsar, etc. 
        The figure \ref{fig:p-pdot} is an example of  
        P-$\dot{\mathrm{P}}$ diagram. The horizontal axis is pulsars' rotation periods and the vertical axis is 
        time derivative of rotation periods.
        \begin{figure}[h]
            \centering
            \includegraphics[scale=0.35]{{/home/wwc129/Desktop/Thesis/ppdot.png}}
            \caption{\textit{\footnotesize Position of pulsars in P-}\footnotesize$\dot{P}$ \textit{\footnotesize
                        diagram}}
            \label{fig:p-pdot}
        \end{figure}
        In this P-$\dot{\mathrm{P}}$ diagram, the negative slope lines represent the strength of surface 
        magnetic field while the positive slope lines represent the characteristic age of pulsars. The 
        following is a short explanation for this. From previous discussion, we have known that the 
        characteristic age of a pulsar is $\tau=-P/\dot{P}=P/(-\dot{P})$, so line of constant 
        $\tau$ is a set of straight lines with equal positive slope. We also know $B\propto\sqrt{P\dot{P}}$,
        therefore the line of constant $B$ should be a part of hyperbola. When $\dot{P}$ is very small, the
        hyperbola looks like a straight line with negative slope. \\
        \indent 
        This figure shows that most pulsars lie in the position about $1s, 10^{-14}s/s$.
        At the same time, a couple of stars lie at the bottom-left of the figure---these are millisecond
        pulsars (MSP). Their 
        rotation periods are about 1-20 milliseconds. It is believed that MSPs are spun up by accretion of 
        mass from its 
        companion star. In the above P-$\dot{\mathrm{P}}$ diagram, we can observe that millisecond pulsars' 
        surface magnetic field are about 3 to 4 orders of magnitude lower than those of normal pulsars. However,
        an MSP has a relative strong magnetic field near its light cylinder. The reason is that an MSP's radius 
        of light cylinder ($R_{lc}=c/\omega)$ is much smaller than a normal pulsar's because of its short 
        rotation period and the magnetic field near light cylinder can be estimated as 
        $B_{lc}\sim\left(R/R_{lc}\right)^3$. At the same time, pulsars' emission mechanism is closely related 
        to their magnetic field near light cylinder. As a result, like a normal pulsar, an MSP also have 
        broadband spectrum from radio to gamma rays. 
        \subsection{Origin Of Millisecond Pulsars}
            From pulsars' emission mechanism, we know that magnetic field of a pulsar decreases with time while 
            the spin period increase with time. But MSPs' spin period is much shorter than normal pulsars and 
            surface magnetic field is a lot weaker. This makes an MSP seem to be both young and old. As a result,
            people think millisecond pulsars are old pulsars spun up by its companion. The companion star transfer
            mass and angular momentum to accelerate the pulsar. Therefore, the aged pulsar can spin faster 
            gradually. 
            \subsubsection{Mass Transfer And Accretion In Binary Systems}
                X-ray binaries are a type of binary systems that is luminous in X-ray band. There are several kinds 
                of X-ray binaries including low mass X-ray binaries (LMXB) and high mass X-ray binaries (HMXB). 
                The way of transferring mass is different in these two types of systems. Before discussing mass 
                transfer, we need to know a little bit about Roche Lobe. The figure \ref{fig:roche lobe} is a 
                schematic diagram of Roche lobe.
                \begin{figure}[h]
                  \centering
                  \includegraphics[scale=0.5]{/home/wwc129/Desktop/Thesis/roche_lobes.jpg}
                  \begin{minipage}{0.8\textwidth}
                  \caption{\footnotesize \textit{Schematic diagram of Roche lobe.} $L_{1}$ is called inner 
                            Lagrange point which is the intersection of equipotential lines of star A 
                            and B.}
                  \label{fig:roche lobe}
                  \end{minipage}
                \end{figure}\\
                \indent  We call two stars in an LMXB as A and B respectively for convenience. It is obvious 
                that if an object is close to star A, the gravitational influence of A is so strong that we can 
                nearly ignore the effect of star B. Similarly, this is true for star B. As a result, there must be 
                a point where the effect of star A is equal to star B which is called inner Lagrange point 
                \hypersetup{urlcolor= red}
                (\href{https://arxiv.org/pdf/astro-ph/0311272.pdf}{\textit{Seidov 2013}}). The two
                volumes inside the largest equipotential lines of A and B are called Roche lobe. If star B cross 
                its Roche lobe, than its mass will be attracted by A thus mass transfer between A and B happens. We 
                should notice that this is the main way of mass transfer in LMXB. While in HMXB, the mass can be 
                transferred by strong wind of the massive companion star. \\
                \indent 
                What should be noted is that mass transfer can change the distance between two companion stars. If 
                low-mass star transfer mass to high-mass companion star, the orbital separation will be large. 
                This can actually stop mass transfer and is like negative feedback. 
                On the contrary, mass transfer from high-mass star to low-mass star will shrink the orbital 
                distance.\\
                
                
                



                
        \subsection{Class II MSPs}
            \begin{figure}[h!]   
                \centering
                \includegraphics[width=6.7cm,height=7cm]{{/home/wwc129/Desktop/Thesis/bands.png}}
                \caption{\textit{\footnotesize Pulse profiles of PSR B1937+21 in radio, X-ray and gamma-ray.}}
                \label{fig:class }
            \end{figure}	 
            Radio emission are usually considered to be emitted above the polar cap, which means radio emission 
            and gamma-ray emission are from different location of pulsar's magnetosphere. However, there are about 
            10 sources showing aligned pulse profiles in radio and gamma-ray implying that radio emission may 
            produced from outer magnetosphere and they are called Class II MSPs\hypersetup{urlcolor= red}
            (\href{https://arxiv.org/pdf/1110.1271.pdf}{\textit{Guillemot et al. 2012}}). 
			These pulsars have strong magnetic 
            field near the light cylinder. The figure \ref{fig:class } is an example of aligned pulse profile.

    \section{Objectives}
    Recently, it is found that X-ray band of spectrum of millisecond pulsar B1937+21 has a hard photon 
    index of $0.9\mathrm{\pm}0.1$ by analysing data of \textit{Chandra}, \textit{XMM-Newton} \textit{Fermi} 
    (\href{https://arxiv.org/pdf/1110.1271.pdf}{\textit{Ng et al. 2014}}).
    People think X-ray emission
    is mainly produced by synchrotron radiation, but it is difficult to explain such a hard photon index. 
    Besides millisecond pulsar B1937+21, it is discovered that hard photon index is a common characteristic 
    of class II MSPs. Therefore, people propose that Inverse Compton scattering can also lead to X-ray 
    emission and build a model recently to explain it (\href{https://arxiv.org/pdf/1110.1271.pdf}
    {\textit{Ng et al. 2014}}). \\
    \indent 
    In order to test the model, we need to analyse the hard X-ray band of millisecond pulsars B1937+21, 
    J0218+4232 and B1821-24. The energy ranges of \textit{Chandra} and \textit{XMM-Newton} are up to 10keV 
    and 15keV respectively while \textit{NuSTAR} can be up to 79keV. As a result, we aim to use
    \textit{NuSTAR} to measure the hard X-ray band of the three MSPs mentioned above. 



    \section{NuSTAR}  
        \textit{NuSTAR} stands for Nuclear Spectroscopic Telescope Array and is launched in June 12, 2012. It is the 
        first space telescope
        focusing on hard X-ray (3eV-79eV) band. This is the telescope that we mainly use and it is helpful to know 
        the structure to deal
        with its data. It mainly composed of three parts: detectors, optics and mast as the figure \ref{fig:nustar}
        shows.
        \begin{figure}[h] 
            \centering
            \includegraphics[scale=0.6]{{/home/wwc129/Desktop/Thesis/nustar.png}}
            \caption{\textit{\footnotesize \textit{NuSTAR}'s sketch. The mast connects X-ray optics and focal plane 
            detectors.}}
            \label{fig:nustar}
        \end{figure}
              
        \subsection{Detectors} 
            \textit{NuSTAR} has two independent photons counting detector modules (FPMA \& FPMB) and each module 
            contains 4 
            Cadmium-Zinc-Telluride (CZT) detectors. Every detector is a rectangular crystal whose size is 
            $20mm\times 20mm 
            \times 2mm$ (length$\times$ width$\times$height) and have $32\times 32$ pixels. \\
            \begin{figure}[h]  
                \hspace{2.3cm} 
                \begin{minipage}[c]{0.4\textwidth}
                    \includegraphics[width=5cm,height=5cm]{{/home/wwc129/Desktop/Thesis/detector.png}}
                \end{minipage}
                \hspace{0.5cm} 
                \begin{minipage}[c]{0.4\textwidth}
                    \includegraphics[width=5cm,height=5cm]{{/home/wwc129/Desktop/Thesis/sheld.png}}
                \end{minipage}
                \newline
                \noindent \hspace*{0.08\linewidth}
                \begin{minipage}[c]{0.8\linewidth}
                \caption{\textit{\footnotesize Left: One of two detector modules which contains $2\times 2$ array 
                of independent detectors.
                Right: One detector module shielded by Csl crystal. CZT detectors can turn high 
                energy photons into electrons very efficiently in room temperature so they are operated at 
                15\textdegree{}C. }}
                \end{minipage}
                \label{fig:detectors}
            \end{figure}
            %\vspace{0.35cm}
            \indent In order to help to distinguish the source photons and the background photons, the focal 
            planes are shielded with 
            crystals made of Celsium-Iodide (Csl). The Csl shields can record the photons come from directions 
            which are not the 
            direction of \textit{NuSTAR} optical axis. So background photons can be subtracted from the total photon 
            counts. 
              
        \subsection{Optics}
            Corresponding to two detector modules, \textit{NuSTAR} also has two optics called Optics Module A and B 
            (OMA \& OMB). The 
            focal length is 10.14 meters which is about the same length with its mast. X-ray is very hard to 
            reflect so mirrors are 
            usually made of high density materials such as Pt and W. Past telescopes such as \textit{Chandra} uses 
            these high density 
            materials to reflect low energy X-ray (up to 10eV). However, the efficiency of reflecting high 
            energy X-ray drops 
            drastically. High density contrast between two kinds of materials are needed to overcome this
            problem. As a 
            result, \textit{NuSTAR}'s mirror is coated with Pt/SiC and W/Si multilayers and can reflect hard X-ray 
            up to 79eV. \\
            \indent Besides high density contrast between two materials, a small incident angle is also 
            required. As the figure \ref{fig:optics}
            \begin{figure}[h]
                \centering
                \includegraphics[scale=0.6]{{/home/wwc129/Desktop/Thesis/lightscheme.png}}
                \caption{\textit{\footnotesize Light path schematic diagram of reflecting X-ray}}
                \label{fig:optics}
            \end{figure}
            showing, the focal length may be very long because of the small incidence angle. This is partly 
            the reason why
            \textit{NuSTAR}'s detectors and optics are separated by a 10-meter long mast which will be introduced in 
            the next section.
			
        \subsection{Mast}
            Although \textit{NuSTAR}'s mast is stable and reliable, it can cause some image distortion because of 
            its deformation. Therefore, 
            careful calibration or measurement of mast's deformation is necessary. In order to achieve this, 
            \textit{NuSTAR} has a laser 
            metrology system which consists of two lasers and two light-sensing detectors. The two 
            lasers are located on optics 
            while the two detectors are mounted on the detector module. Then, the deformation can be recorded 
            and used to reconstruct the raw data. The figure \ref{fig:mast} shows what the mast looks like. The
            reason why \textit{NuSTAR} has a deployable mast is that it is carried by a relative small rockets.
            \begin{figure}[h!]
              \centering
              \includegraphics[scale=0.35]{/home/wwc129/Desktop/Thesis/nustar_mast.png}
              \mycaption{\textit{NuSTAR}'s mast. Left: stored in container. Right: after being deployed.}
              \label{fig:mast}
            \end{figure}

        \subsection{Performance of NuSTAR}
            Though \textit{NuSTAR} has a broad energy range, the effective collecting area at different energy is 
            quite different. The figure \ref{fig:effective_area} shows comparison between \textit{NuSTAR} and other 
            telescopes.
            \begin{figure}[h!]
              \centering
              \includegraphics[scale=0.35]{/home/wwc129/Desktop/Thesis/effect_area.png}
              \mycaption{\textit{NuSTAR}'s effective area compared with other X-ray focusing telescopes.}
              \label{fig:effective_area}
            \end{figure}
            
            From figure \ref{fig:effective_area} we can see that the effective area drops dramatically after 70keV.
            Therefore, we may need to screen out the high energy part ($>$70keV) for data analysis.
            The figure \ref{fig:psf} shows the point spread function (PSF) of optics module A and B. In order to
            make faint pixels look more obvious, the images are in logarithm scale. The PSF of both optics module A
            and B are dependent on energy. The table \ref{table:psf_relation} lists the relationships and from this 
            table we can also see that angular resolution of optics module B is slightly better A.
            
             
            \begin{figure}[!htp]
              \centering
              \includegraphics[scale=0.27]{/home/wwc129/Desktop/Thesis/psf.png}
              \mycaption{Image of NuSTAR's point spread function of optics module A(left) and B(right).}
              \label{fig:psf}
              \vspace{1.5cm} 
              \includegraphics[scale=0.35]{/home/wwc129/Desktop/Thesis/psf_relation.png}
              \mycaption{PSF (half power diameter) as a function of energy.}
              \label{table:psf_relation}
            \end{figure}

        \section{The Procedures of Processing NuSTAR Data}
            \textit{NuSTAR} Data Analysis Software (\textit{NuSTARDAS}) is used for data processing. This includes 
            three steps:
            data calibration, data screening and products extraction. The next paragraph talks about the reason the 
            first two steps are necessary. \\
            \indent 
            Lots of factors can affect raw data, such as movements of \textit{NuSTAR}‘s mast, the orbit of the 
            telescope, 
            etc. As a result, \textit{NuSTAR} data must be calibrated before they can be used to do data analysis. 
            Also, some parts 
            of data are not good for scientific analysing. For example, there are some bad pixels in the telescope's 
            detectors which cannot record photons correctly. Thus, the data recorded by these bad pixels have to be
            treated very carefully. We might get rid of the data when the source is bright, while if the photon counts
            are too small, we might have to use these data with carefulness. And sometimes we may want to focus on a 
            particular part of data. Thus, we need to screen the calibrated data. \\
            \indent 
            In order to understand data filter process better, it is necessary to know the different levels of 
            \textit{NuSTAR} data. 
            \textit{NuSTAR} data can be divide into 5 levels which is from level 0 to level 3. Level 0 data are raw 
            telemetry files
            which might not be in formal format (FITS format). Level 1 data contains two parts: level 1 and level 
            1a. Level 1
            data are formatted in FITS format, but not calibrated yet. Level 1a data are level 1 data after 
            calibration. 
            Actually, level 1a data are addition of level 1 data and calibration data. Level 1a data are produced by 
            step 1 
            (data calibration). Then by step 2 (data screening), we get level 2 data which are cleaned files. We 
            can do data
            analysis after getting level 2 data. Thus in order to get reliable result, it is key to get raw data 
            properly 
            cleaned. The figure \ref{fig:comparison} is comparison between level 1a event file and level 2 event file 
            after screening.  

            \begin{figure}[h] 
            \vspace{0.2cm}
                \begin{minipage}{0.45\textwidth}
                \begin{flushright} 
                \includegraphics[scale=0.31]{/home/wwc129/Desktop/Thesis/A_uf.png}
             %\caption{raw picture}
            \end{flushright}
            \end{minipage}
            \hspace{1cm}
            \begin{minipage}{0.45\textwidth}
              \begin{flushleft}
                \includegraphics[scale=0.35]{/home/wwc129/Desktop/Thesis/A_cl.png}
              \end{flushleft} 
            \end{minipage}
            \centering
            \begin{minipage}[c]{0.85\textwidth}
                \caption{\textit{\footnotesize Left: sky image generated directly from level 1a data. Right: sky 
                image 
                generated from screened level 2 data. These two figures are produced from the same raw data. 
                There are 
                many criteria for data screening and the figure in the right is just an example.}}
                \label{fig:comparison}
            \end{minipage}
            \end{figure}
            
            \subsection{Data Calibration}   
                The first process is data calibration. In this step, two factors should be considered: the temporal 
                change 
                of mast and spacecraft's attitude. By using telescope's housekeeping files, the corrected data can be 
                produced by \textit{NuSTAR} software. There are some \textit{NuSTAR} software modules for this 
                process. Most of them are 
                nearly automatically, which means once we have initial files, the output files are fixed. 
                Therefore, we 
                do not concentrate much on these software modules. Before we go into this module, some basic 
                concepts should be introduced.

            \subsubsection{Grade of Data}  
            When a photon interacts with a detector, the ideal condition is that only one pixel record this photon,
            which has better spectra resolution than other situations. 
            But a single X-ray photon can be spread and received by more than one pixel. Generally, the less pixels
            by which a photon recorded, the better the event is. 
            Naturally, there are different patterns of interaction between the
            photon and surrounding pixels. These patterns are listed in figure \ref{fig:nustar_grade}.  
            \begin{figure}[h] 
              \centering
              \includegraphics[scale=0.34]{/home/wwc129/Desktop/Thesis/grade.png}
              \begin{minipage}[c]{0.85\textwidth}
                \caption{\textit{\footnotesize There are 33 different \textit{NuSTAR} grades---from grade 0 to 32.
                          In this figure, grades from 0 to 26 are listed because these grades are accepted by 
                          \textit{NuSTAR} data screening by default. We can further get rid of some grades if needed.}
                          }
              \label{fig:nustar_grade}
              \end{minipage}
              
            \end{figure}


            \subsubsection{Status of Data} 
            Just like flagging the quality of pixel data, it is necessary to flag event data because there are many
            factors that can trigger detectors. For instance, if we have a photon record in a detector, we need to 
            know if it is from the source we are observing. Actually, cosmic rays can also trigger the 
            detectors and we want to get rid of them in order to increase the signal to noise ratio. 
            Therefore, after we have distinguished them from source photons, it is needed to be recorded for data
            screening---thus each single event has its status. In fact, similar to data's grade, status of data also
            has many patterns. For example, an event may fall into bad pixels, have a neighborhood bad pixel
            or fall into a hot pixel, etc. All these different situations are recorded by a 16-bit binary number and
            status of good event which is ideal for scientific analysis is all zero (recorded as "b0000000000000000").

            
            \subsection{Data Screening}  
              After data calibration, though we know if one event is good or not, bad events are not excluded from 
              original
              files. This is the primary reason of doing data screening. There are primarily 3 procedures in this 
              step.\\
              \indent The first one is choosing good time intervals, which means remove some unwanted time 
              intervals. For 
              example, when the telescope is in the South Atlantic Anomaly, when the Earth is in the Field Of 
              View and 
              when the motion of mast is not well tracked, etc. Sometimes we need to add our own GTI (Good Time 
              Interval) file to get better cleaned data. 
              Then remove bad pixels and events flagged as 
              bad in the 
              last step (by using the information of data status). At last, choose the proper grade of data 
              (the default value is 0-26). \\
              \indent The core module in this step is called 'nuscreen' and most of job is done by this software 
              module.  
              There are two parameters we mostly focus on---'gradeexpr' and 'statusexpr', which are used for choosing 
              grade of data and status of data respectively. For example, gradeexpr=0-8 means choosing grade range 
              from 0 to 8 and statusexpr="STATUS==b000000000x0xx000" means leaving out bad events. We also can set  
              them to default value by 'statusexpr=DEFAULT' and 'statusexpr=DEFAULT'. By adjusting these parameters' 
              values, we can get a bunch of cleaned event files. Then we can generate several sky images and 
              spectra and 
              choose a better result. 

             \subsection{Products Extraction}
              The aim of this process is to extract high-level scientific products including light curves, sky images,
              spectra, Ancillary Response Files (ARF) and Redistribution Matrix Files (RMF). ARF and RMF files are 
              used for spectra analysis. The main software module is 'nuproducts' which generates these files 
              automatically by 
              passing some parameters. I mainly focus on two parameters which are 'pilow' and 'pihigh'. These two 
              parameters filter the energy range of cleaned stage 2 event files. The default values of the two 
              parameters
              are 39 Pi and 1909 Pi respectively corresponding the \textit{NuSTAR}'s energy range 3-78.4eV. But 
              usually we 
              choose pilow larger than 39 and pihigh smaller than 1909 in order to get more reliable data.





            



            
           

\chapter{Data Analysis}
    
    
    \section{Data Analysis of Pulsar B1937+21 Using NuSTAR}
        Usually, we choose all the parameters to be default values. We may change some of the parameters in certain 
        cases. 
        Then we can produce a complete set of stage 2 files by using the module 'nupipeline'. The following is 
        the sky images from \textit{NuSTAR}'s module A and module B. 
        \begin{figure}[h]
          \hspace{0.7cm}
          \begin{minipage}{0.45\textwidth} 
            \centering 
            \includegraphics[scale=0.35]{/home/wwc129/Desktop/Thesis/A01.png}
            \caption{\textit{\footnotesize Sky image of module A}}
          \end{minipage}
          \hspace{0.1cm} 
          \begin{minipage}{0.45\textwidth}
            \centering 
            \includegraphics[scale=0.33]{/home/wwc129/Desktop/Thesis/B01.png}
            \caption{\textit{\footnotesize Sky image of module B}}
          \end{minipage}
        \end{figure}
      
        The total events in the left figure is 33533 while in the right figure the number is 28806 
        (including background counts). We can also see that the left figure is a little bit brighter than the figure
        in the right. Although there are lots of background noises in these two pictures, 
        the source is clearly identifiable. 
        The focal plane module comprises 4 detectors, but in these figures we cannot distinguish different detectors.
        This implies that the data are not calibrated and screened well enough. \\
        \indent After getting cleaned stage 2 files, we can continue to generate light curves, spectra and so on. 
        Before doing these, it is required to choose source and background regions. The figure \ref{fig:a1} and 
        \ref{fig:b1} show the regions we choose (this was the region I chose first time). 
        \begin{figure}[h]
          \hspace{0.7cm}
          \begin{minipage}{0.45\textwidth} 
            \centering 
            \includegraphics[scale=0.33]{/home/wwc129/Desktop/Thesis/A01_region.png}
            \caption{\textit{\footnotesize Source and background regions of module A. }}
            \label{fig:a1}
          \end{minipage}
          \hspace{0.1cm} 
          \begin{minipage}{0.45\textwidth}
            \centering 
            \includegraphics[scale=0.344]{/home/wwc129/Desktop/Thesis/B01_region.png}
            \caption{\textit{\footnotesize Source and background regions of module B. }}
            \label{fig:b1}
          \end{minipage}
        \end{figure}
        In the above figures, the left green circle is source region while the right green circle is 
        background region. The center of source circle is (19:39:38.561 ra, 21:34:59.081 dec) which is pretty 
        close to the coordinates of pulsar B1937+21. This means that the observation data can be used to analyse 
        the pulsar B1937+21. Then use \textit{NuSTAR} software module 'nuproducts' to generate sky images, light 
        curves, spectra, source ARF and RMF file (these files are used to do spectra analysis). The figure 
        \ref{fig:sky1} is sky image of module A and B produced by 'nuproducts'.\\
        \begin{figure}[h!]
          \begin{minipage}{0.45\textwidth}
            \begin{flushright}
                \includegraphics[scale=0.3]{/home/wwc129/Desktop/Thesis/A_pro.png}
            \end{flushright}
          \end{minipage}
        \hspace{1cm}
          \begin{minipage}{0.45\textwidth}
            \begin{flushleft}
                \includegraphics[scale=0.35]{/home/wwc129/Desktop/Thesis/B_pro.png}
            \end{flushleft}
          \end{minipage}
          \centering 
          \begin{minipage}[b]{0.85\textwidth}
            \vspace{0.25cm}
            \mycaption{Stage 3 sky images of module A (left) and B (right). The energy range we chose is from 
                        35pi to 1909pi which is the default value.}
            \label{fig:sky1}
          \end{minipage}
          
        \end{figure}
        At first glance, there is no difference between stage 2 sky images and stage 3 sky images. But actually, 
        the pixels of stage 3 sky images are much less. Take module A for example, the dimensions of stage 2 and 3
        sky images are 54137$\times$54137 and 1000$\times$1000. This shows that even though we choose the largest
        reasonable energy range (35-1909pi corresponding to 3-79eV which is the energy range of \textit{NuSTAR}), 
        there are lots of data has been filtered out.\\
        \indent
        The figure \ref{fig:lightcurves} is the light curve of module A and B. Although these pictures show a little 
        bit periodicity, they still look messy. The reason is that the time resolution of \textit{NuSTAR} is 2$ms$
        which is larger than the spin period of B1937+21 ($\sim$1.56ms). As a result, it is not appropriate to do 
        timing analysis of the pulsar B1937+21 using \textit{NuSTAR}. \\
        \begin{figure}[h]
          \begin{minipage}{0.45\textwidth}
            \begin{flushleft}
                \includegraphics[scale=0.31]{/home/wwc129/Desktop/Thesis/nu30101031002A01_lc.png}
              \end{flushleft}
            \end{minipage}
          \hspace{0.1cm} 
          \begin{minipage}{0.45\textwidth}
            \includegraphics[scale=0.31]{/home/wwc129/Desktop/Thesis/nu30101031002B01_lc.png}
          \end{minipage}
          \mycaption{Light curves of module A (left) and B (right).}
          \label{fig:lightcurves}
        \end{figure}

        The figure \ref{spectra} is the spectra of module A and B. The group counts of both module A and B are
        20 and the fitting model is multiplication of absorption of X-ray model (see in sherpa document 
        \href{http://%
        cxc.harvard.edu/sherpa/ahelp/xstbabs.html}{'xstbabs'}) and 1 dimensional power law model (see also in sherpa 
        document\href{http://cxc.harvard.edu/sherpa/ahelp/powlaw1d.html}{'powlaw1d'}). We use chi square statistics
        to analyse goodness of fitting.\\
        \indent
        The number of total points in spectrum of module A is 17 and B is 15. For the spectrum of A, the 
        reduced statistic is 0.67058 and Q-value is 0.805413. For the spectrum of B, the reduced statistic is 0.48646
        and Q-value is 0.92405. \newpage 
        \begin{figure}[!ht]
          \begin{minipage}[c]{0.45\textwidth}
            \begin{flushleft} 
                \includegraphics[scale=0.45]{/home/wwc129/Desktop/Thesis/spectrum_a.png}
            \end{flushleft}
            \end{minipage}
          \begin{minipage}{0.45\textwidth}
            \begin{flushleft}
            \includegraphics[scale=0.45]{/home/wwc129/Desktop/Thesis/spectrum_b.png}
            \end{flushleft}
          \end{minipage}
          \centering
          \begin{minipage}{0.8\textwidth}
          \mycaption{Spectra of module A (left) and B (right). The total counts of A before grouping is 
                      255 and B is 209 (after subtracting background counts).}
          \label{spectra}
          \end{minipage}
          \end{figure}
        













		
			

    



\end{document}
























