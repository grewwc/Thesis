\documentclass[12pt]{report}
\usepackage{geometry}	
\usepackage[utf8]{inputenc}
\usepackage{amsmath}
\usepackage{multicol}
\usepackage{titlesec}
\usepackage{graphicx}
\usepackage{wrapfig}
\usepackage{textcomp}
\usepackage{caption}
\usepackage{subcaption}
\usepackage{comment}
\usepackage{etoolbox}
\usepackage{anyfontsize}
\usepackage{url}
\usepackage{multirow}
\usepackage{array}
\usepackage{tabu}
\usepackage{hyperref}
\usepackage{color}
\usepackage{epigraph}
\usepackage{makebox}
\usepackage{graphicx}
\usepackage{array}
\usepackage{setspace}
\usepackage{natbib}
\usepackage{mathrsfs}
\usepackage{caption}
\usepackage{makecell}
\geometry{
	a4paper,
	% total={210mm,297mm},
 	left=35mm,
 	top=25mm,
 	right=35mm,
}

\graphicspath{{C:/Users/User/Desktop/Thesis/}{/Users/grewwc/Desktop/Thesis/}}

\captionsetup{font={stretch=1.5}}

\begin{document}

  \begin{center}
    \Large \textbf{Abstract}
  \end{center}
  \doublespacing
  PSRs J0218+4232, B1821$-$24 and B1937+21 are among the most energetic and fastest-spinning 
  millisecond pulsars (MSPs). They have been studied in all radio, X-ray and gamma-ray bands. 
  The \textit{Fermi} LAT Pass 8 data was published in 2015 and has lots of advantages over 
  the old Pass 7 data, such as increased effective area and wider energy range. Since 
  the recent gamma-ray spectral analysis of the three MSPs are relatively old, 
  I redo the gamma-ray spectral analysis of the MSPs with 
  four-year more \textit{Fermi} LAT observational data and newly published Pass 8 data. 
  I obtain better fit results for gamma-ray spectra of the three MSPs with smaller errors 
  and larger test statistic values. I also do numerical simulations to test the 
  two-layer model using the new observational data. By minimizing the differences between 
  the predictions of the two-layer model and the real data, I fit the independent 
  parameters of the two-layer model. I find that the simplified two-layer model can 
  predict broadband spectra of the three MSPs which are very close to the observational data 
  from \textit{Fermi} LAT in most energy ranges.

  \vspace{2cm}
  \hspace*{10cm} Word-count: $183$
  
\end{document}




