\documentclass[12pt]{report}
\usepackage{geometry}	
\usepackage[utf8]{inputenc}
\usepackage{amsmath}
\usepackage{multicol}
\usepackage{titlesec}
\usepackage{graphicx}
\usepackage{wrapfig}
\usepackage{textcomp}
\usepackage{caption}
\usepackage{subcaption}
\usepackage{comment}
\usepackage{etoolbox}
\usepackage{anyfontsize}
\usepackage{url}
\usepackage{multirow}
\usepackage{array}
\usepackage{tabu}
\usepackage{hyperref}
\usepackage{color}
\usepackage{epigraph}
\usepackage{makebox}
\usepackage{graphicx}
\usepackage{array}
\usepackage{setspace}
\usepackage{natbib}
\usepackage{mathrsfs}
\usepackage{caption}
\usepackage{makecell}
\pagenumbering{gobble}
\geometry{
	a4paper,
	% total={210mm,297mm},
 	left=35mm,
 	top=25mm,
 	right=35mm,
}

\graphicspath{{C:/Users/User/Desktop/Thesis/}{/Users/grewwc/Desktop/Thesis/}}

\captionsetup{font={stretch=1.5}}

\begin{document}

\begin{center}
  \vspace*{3cm}
  % Gamma-ray Spectral Analysis of Three Energetic Millisecond Pulsars
  % \Large \textbf{GAMMA-RAY SPECTRAL ANALYSIS 
  %   \\[0.5cm] OF THREE ENERGETIC \\ [0.5cm] 
  %   MILLISECOND PULSARS \\[3.7cm]}
  % \small
  %   A THESIS \\[0.5cm] 
  %   SUBMITTED TO THE DEPARTMENT OF PHYSICS \\[0.5cm]
  %   OF THE UNIVERSITY OF HONG KONG \\[0.5CM]
  %   IN PARTIAL FULFILLMENT OF THE REQUIREMENTS \\[0.5cm]
  %   FOR THE DEGREE OF \\[0.5cm]
  %   MASTER OF PHILOSOPHY \\[4.2cm]

  %   By \\[0.5cm]
  %   \large Wenchao Wang \\[0.5cm]
  %   August 2018
  %   \newpage 
    Abstract of thesis entitled \\[1.2cm]
    \Large \textbf{GAMMA-RAY SPECTRAL ANALYSIS 
    \\[0.5cm] OF THREE ENERGETIC \\ [0.5cm] 
    MILLISECOND PULSARS \\[0.5cm]}
    \normalsize Submitted by \\ [0.5cm]
    \Large \textbf{Wenchao Wang} \\[1cm]
    \normalsize for the degree of Master of Philosophy \\[0.5cm]
    at the University of Hong Kong \\[0.5cm]
    in August 2018 \\[1.9cm]
  \end{center}


  \doublespacing
  A millisecond pulsar (MSP) is a fast-spinning pulsar whose rotational period is a few 
  milliseconds. MSPs are believed to be old pulsars spun up by their companion stars. PSRs
  J0218+4232, B1821$-$24 and B1937+21 are among the most energetic and fastest-spinning  
  % J0218+4232, B1821−24 and B1937+21 are among the most energetic and fastest-spinning 
  MSPs. They have been studied in radio, X-ray and gamma-ray bands, and show 
  aligned pulse profile in different energy bands. However, all previous gamma-ray studies 
  were done with previous Fermi LAT Pass 7 data or earlier. The Fermi LAT Pass 8 data was 
  published in 2015 and has substantial improvements, such as increased effective area and 
  wider energy range. Since the recent gamma-ray spectral analyses of the three MSPs are 
  relatively old, I re-analyzed the gamma-ray spectra of the three MSPs with four-year more 
  Fermi LAT observational data and newly published Pass 8 data. Additionally, new X-ray 
  studies of the three MSPs using NuSTAR had been published. I obtained better fit results
  for gamma-ray spectra of the three MSPs with smaller error bars and larger test statistic 
  values. I built a numerical model to explain the high-energy emission from X-rays to 
  gamma-rays based on a two-layer outer gap model. By minimizing the differences between 
  the predictions of the two-layer model and the real data, I fitted three independent 
  parameters of the model. It is found that the simplified two-layer model can predict 
  broadband spectra of the three MSPs which are very close to the observational data from 
  in both X-rays and gamma-rays.

  % This leads me to study more thorough 


  \vspace{4.2cm}
  \hspace*{10cm} Word-count: $243$
  
\end{document}




